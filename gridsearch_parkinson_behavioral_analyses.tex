% Options for packages loaded elsewhere
\PassOptionsToPackage{unicode}{hyperref}
\PassOptionsToPackage{hyphens}{url}
\PassOptionsToPackage{dvipsnames,svgnames,x11names}{xcolor}
%
\documentclass[
  letterpaper,
  DIV=11,
  numbers=noendperiod]{scrreprt}

\usepackage{amsmath,amssymb}
\usepackage{iftex}
\ifPDFTeX
  \usepackage[T1]{fontenc}
  \usepackage[utf8]{inputenc}
  \usepackage{textcomp} % provide euro and other symbols
\else % if luatex or xetex
  \usepackage{unicode-math}
  \defaultfontfeatures{Scale=MatchLowercase}
  \defaultfontfeatures[\rmfamily]{Ligatures=TeX,Scale=1}
\fi
\usepackage{lmodern}
\ifPDFTeX\else  
    % xetex/luatex font selection
\fi
% Use upquote if available, for straight quotes in verbatim environments
\IfFileExists{upquote.sty}{\usepackage{upquote}}{}
\IfFileExists{microtype.sty}{% use microtype if available
  \usepackage[]{microtype}
  \UseMicrotypeSet[protrusion]{basicmath} % disable protrusion for tt fonts
}{}
\makeatletter
\@ifundefined{KOMAClassName}{% if non-KOMA class
  \IfFileExists{parskip.sty}{%
    \usepackage{parskip}
  }{% else
    \setlength{\parindent}{0pt}
    \setlength{\parskip}{6pt plus 2pt minus 1pt}}
}{% if KOMA class
  \KOMAoptions{parskip=half}}
\makeatother
\usepackage{xcolor}
\setlength{\emergencystretch}{3em} % prevent overfull lines
\setcounter{secnumdepth}{5}
% Make \paragraph and \subparagraph free-standing
\makeatletter
\ifx\paragraph\undefined\else
  \let\oldparagraph\paragraph
  \renewcommand{\paragraph}{
    \@ifstar
      \xxxParagraphStar
      \xxxParagraphNoStar
  }
  \newcommand{\xxxParagraphStar}[1]{\oldparagraph*{#1}\mbox{}}
  \newcommand{\xxxParagraphNoStar}[1]{\oldparagraph{#1}\mbox{}}
\fi
\ifx\subparagraph\undefined\else
  \let\oldsubparagraph\subparagraph
  \renewcommand{\subparagraph}{
    \@ifstar
      \xxxSubParagraphStar
      \xxxSubParagraphNoStar
  }
  \newcommand{\xxxSubParagraphStar}[1]{\oldsubparagraph*{#1}\mbox{}}
  \newcommand{\xxxSubParagraphNoStar}[1]{\oldsubparagraph{#1}\mbox{}}
\fi
\makeatother

\usepackage{color}
\usepackage{fancyvrb}
\newcommand{\VerbBar}{|}
\newcommand{\VERB}{\Verb[commandchars=\\\{\}]}
\DefineVerbatimEnvironment{Highlighting}{Verbatim}{commandchars=\\\{\}}
% Add ',fontsize=\small' for more characters per line
\usepackage{framed}
\definecolor{shadecolor}{RGB}{241,243,245}
\newenvironment{Shaded}{\begin{snugshade}}{\end{snugshade}}
\newcommand{\AlertTok}[1]{\textcolor[rgb]{0.68,0.00,0.00}{#1}}
\newcommand{\AnnotationTok}[1]{\textcolor[rgb]{0.37,0.37,0.37}{#1}}
\newcommand{\AttributeTok}[1]{\textcolor[rgb]{0.40,0.45,0.13}{#1}}
\newcommand{\BaseNTok}[1]{\textcolor[rgb]{0.68,0.00,0.00}{#1}}
\newcommand{\BuiltInTok}[1]{\textcolor[rgb]{0.00,0.23,0.31}{#1}}
\newcommand{\CharTok}[1]{\textcolor[rgb]{0.13,0.47,0.30}{#1}}
\newcommand{\CommentTok}[1]{\textcolor[rgb]{0.37,0.37,0.37}{#1}}
\newcommand{\CommentVarTok}[1]{\textcolor[rgb]{0.37,0.37,0.37}{\textit{#1}}}
\newcommand{\ConstantTok}[1]{\textcolor[rgb]{0.56,0.35,0.01}{#1}}
\newcommand{\ControlFlowTok}[1]{\textcolor[rgb]{0.00,0.23,0.31}{\textbf{#1}}}
\newcommand{\DataTypeTok}[1]{\textcolor[rgb]{0.68,0.00,0.00}{#1}}
\newcommand{\DecValTok}[1]{\textcolor[rgb]{0.68,0.00,0.00}{#1}}
\newcommand{\DocumentationTok}[1]{\textcolor[rgb]{0.37,0.37,0.37}{\textit{#1}}}
\newcommand{\ErrorTok}[1]{\textcolor[rgb]{0.68,0.00,0.00}{#1}}
\newcommand{\ExtensionTok}[1]{\textcolor[rgb]{0.00,0.23,0.31}{#1}}
\newcommand{\FloatTok}[1]{\textcolor[rgb]{0.68,0.00,0.00}{#1}}
\newcommand{\FunctionTok}[1]{\textcolor[rgb]{0.28,0.35,0.67}{#1}}
\newcommand{\ImportTok}[1]{\textcolor[rgb]{0.00,0.46,0.62}{#1}}
\newcommand{\InformationTok}[1]{\textcolor[rgb]{0.37,0.37,0.37}{#1}}
\newcommand{\KeywordTok}[1]{\textcolor[rgb]{0.00,0.23,0.31}{\textbf{#1}}}
\newcommand{\NormalTok}[1]{\textcolor[rgb]{0.00,0.23,0.31}{#1}}
\newcommand{\OperatorTok}[1]{\textcolor[rgb]{0.37,0.37,0.37}{#1}}
\newcommand{\OtherTok}[1]{\textcolor[rgb]{0.00,0.23,0.31}{#1}}
\newcommand{\PreprocessorTok}[1]{\textcolor[rgb]{0.68,0.00,0.00}{#1}}
\newcommand{\RegionMarkerTok}[1]{\textcolor[rgb]{0.00,0.23,0.31}{#1}}
\newcommand{\SpecialCharTok}[1]{\textcolor[rgb]{0.37,0.37,0.37}{#1}}
\newcommand{\SpecialStringTok}[1]{\textcolor[rgb]{0.13,0.47,0.30}{#1}}
\newcommand{\StringTok}[1]{\textcolor[rgb]{0.13,0.47,0.30}{#1}}
\newcommand{\VariableTok}[1]{\textcolor[rgb]{0.07,0.07,0.07}{#1}}
\newcommand{\VerbatimStringTok}[1]{\textcolor[rgb]{0.13,0.47,0.30}{#1}}
\newcommand{\WarningTok}[1]{\textcolor[rgb]{0.37,0.37,0.37}{\textit{#1}}}

\providecommand{\tightlist}{%
  \setlength{\itemsep}{0pt}\setlength{\parskip}{0pt}}\usepackage{longtable,booktabs,array}
\usepackage{calc} % for calculating minipage widths
% Correct order of tables after \paragraph or \subparagraph
\usepackage{etoolbox}
\makeatletter
\patchcmd\longtable{\par}{\if@noskipsec\mbox{}\fi\par}{}{}
\makeatother
% Allow footnotes in longtable head/foot
\IfFileExists{footnotehyper.sty}{\usepackage{footnotehyper}}{\usepackage{footnote}}
\makesavenoteenv{longtable}
\usepackage{graphicx}
\makeatletter
\newsavebox\pandoc@box
\newcommand*\pandocbounded[1]{% scales image to fit in text height/width
  \sbox\pandoc@box{#1}%
  \Gscale@div\@tempa{\textheight}{\dimexpr\ht\pandoc@box+\dp\pandoc@box\relax}%
  \Gscale@div\@tempb{\linewidth}{\wd\pandoc@box}%
  \ifdim\@tempb\p@<\@tempa\p@\let\@tempa\@tempb\fi% select the smaller of both
  \ifdim\@tempa\p@<\p@\scalebox{\@tempa}{\usebox\pandoc@box}%
  \else\usebox{\pandoc@box}%
  \fi%
}
% Set default figure placement to htbp
\def\fps@figure{htbp}
\makeatother
% definitions for citeproc citations
\NewDocumentCommand\citeproctext{}{}
\NewDocumentCommand\citeproc{mm}{%
  \begingroup\def\citeproctext{#2}\cite{#1}\endgroup}
\makeatletter
 % allow citations to break across lines
 \let\@cite@ofmt\@firstofone
 % avoid brackets around text for \cite:
 \def\@biblabel#1{}
 \def\@cite#1#2{{#1\if@tempswa , #2\fi}}
\makeatother
\newlength{\cslhangindent}
\setlength{\cslhangindent}{1.5em}
\newlength{\csllabelwidth}
\setlength{\csllabelwidth}{3em}
\newenvironment{CSLReferences}[2] % #1 hanging-indent, #2 entry-spacing
 {\begin{list}{}{%
  \setlength{\itemindent}{0pt}
  \setlength{\leftmargin}{0pt}
  \setlength{\parsep}{0pt}
  % turn on hanging indent if param 1 is 1
  \ifodd #1
   \setlength{\leftmargin}{\cslhangindent}
   \setlength{\itemindent}{-1\cslhangindent}
  \fi
  % set entry spacing
  \setlength{\itemsep}{#2\baselineskip}}}
 {\end{list}}
\usepackage{calc}
\newcommand{\CSLBlock}[1]{\hfill\break\parbox[t]{\linewidth}{\strut\ignorespaces#1\strut}}
\newcommand{\CSLLeftMargin}[1]{\parbox[t]{\csllabelwidth}{\strut#1\strut}}
\newcommand{\CSLRightInline}[1]{\parbox[t]{\linewidth - \csllabelwidth}{\strut#1\strut}}
\newcommand{\CSLIndent}[1]{\hspace{\cslhangindent}#1}

\usepackage{booktabs}
\usepackage{longtable}
\usepackage{array}
\usepackage{multirow}
\usepackage{wrapfig}
\usepackage{float}
\usepackage{colortbl}
\usepackage{pdflscape}
\usepackage{tabu}
\usepackage{threeparttable}
\usepackage{threeparttablex}
\usepackage[normalem]{ulem}
\usepackage{makecell}
\usepackage{xcolor}
\KOMAoption{captions}{tableheading}
\usepackage{ragged2e}
\makeatletter
\@ifpackageloaded{tcolorbox}{}{\usepackage[skins,breakable]{tcolorbox}}
\@ifpackageloaded{fontawesome5}{}{\usepackage{fontawesome5}}
\definecolor{quarto-callout-color}{HTML}{909090}
\definecolor{quarto-callout-note-color}{HTML}{0758E5}
\definecolor{quarto-callout-important-color}{HTML}{CC1914}
\definecolor{quarto-callout-warning-color}{HTML}{EB9113}
\definecolor{quarto-callout-tip-color}{HTML}{00A047}
\definecolor{quarto-callout-caution-color}{HTML}{FC5300}
\definecolor{quarto-callout-color-frame}{HTML}{acacac}
\definecolor{quarto-callout-note-color-frame}{HTML}{4582ec}
\definecolor{quarto-callout-important-color-frame}{HTML}{d9534f}
\definecolor{quarto-callout-warning-color-frame}{HTML}{f0ad4e}
\definecolor{quarto-callout-tip-color-frame}{HTML}{02b875}
\definecolor{quarto-callout-caution-color-frame}{HTML}{fd7e14}
\makeatother
\makeatletter
\@ifpackageloaded{caption}{}{\usepackage{caption}}
\AtBeginDocument{%
\ifdefined\contentsname
  \renewcommand*\contentsname{Table of contents}
\else
  \newcommand\contentsname{Table of contents}
\fi
\ifdefined\listfigurename
  \renewcommand*\listfigurename{List of Figures}
\else
  \newcommand\listfigurename{List of Figures}
\fi
\ifdefined\listtablename
  \renewcommand*\listtablename{List of Tables}
\else
  \newcommand\listtablename{List of Tables}
\fi
\ifdefined\figurename
  \renewcommand*\figurename{Figure}
\else
  \newcommand\figurename{Figure}
\fi
\ifdefined\tablename
  \renewcommand*\tablename{Table}
\else
  \newcommand\tablename{Table}
\fi
}
\@ifpackageloaded{float}{}{\usepackage{float}}
\floatstyle{ruled}
\@ifundefined{c@chapter}{\newfloat{codelisting}{h}{lop}}{\newfloat{codelisting}{h}{lop}[chapter]}
\floatname{codelisting}{Listing}
\newcommand*\listoflistings{\listof{codelisting}{List of Listings}}
\makeatother
\makeatletter
\makeatother
\makeatletter
\@ifpackageloaded{caption}{}{\usepackage{caption}}
\@ifpackageloaded{subcaption}{}{\usepackage{subcaption}}
\makeatother

\usepackage{bookmark}

\IfFileExists{xurl.sty}{\usepackage{xurl}}{} % add URL line breaks if available
\urlstyle{same} % disable monospaced font for URLs
\hypersetup{
  pdftitle={Exploration and Exploitation in Parkinson's Disease: Behavioral Analyses},
  pdfauthor={Björn Meder; Martha Sterf; Charley M. Wu; Matthias Guggenmos},
  colorlinks=true,
  linkcolor={blue},
  filecolor={Maroon},
  citecolor={Blue},
  urlcolor={Blue},
  pdfcreator={LaTeX via pandoc}}


\title{Exploration and Exploitation in Parkinson's Disease: Behavioral
Analyses}
\author{Björn Meder \and Martha Sterf \and Charley M. Wu \and Matthias
Guggenmos}
\date{2025-02-05}

\begin{document}
\maketitle

\renewcommand*\contentsname{Table of contents}
{
\hypersetup{linkcolor=}
\setcounter{tocdepth}{2}
\tableofcontents
}

\begin{Shaded}
\begin{Highlighting}[]
\CommentTok{\# Housekeeping: Load packages and helper functions}
\CommentTok{\# Housekeeping}
\NormalTok{knitr}\SpecialCharTok{::}\NormalTok{opts\_chunk}\SpecialCharTok{$}\FunctionTok{set}\NormalTok{(}\AttributeTok{echo =} \ConstantTok{TRUE}\NormalTok{)}
\NormalTok{knitr}\SpecialCharTok{::}\NormalTok{opts\_chunk}\SpecialCharTok{$}\FunctionTok{set}\NormalTok{(}\AttributeTok{message =} \ConstantTok{FALSE}\NormalTok{)}
\NormalTok{knitr}\SpecialCharTok{::}\NormalTok{opts\_chunk}\SpecialCharTok{$}\FunctionTok{set}\NormalTok{(}\AttributeTok{warning =} \ConstantTok{FALSE}\NormalTok{)}
\NormalTok{knitr}\SpecialCharTok{::}\NormalTok{opts\_chunk}\SpecialCharTok{$}\FunctionTok{set}\NormalTok{(}\AttributeTok{fig.align=}\StringTok{\textquotesingle{}center\textquotesingle{}}\NormalTok{)}
\NormalTok{knitr}\SpecialCharTok{::}\NormalTok{opts\_chunk}\SpecialCharTok{$}\FunctionTok{set}\NormalTok{(}\AttributeTok{prefer\_html =} \ConstantTok{TRUE}\NormalTok{)}

\FunctionTok{options}\NormalTok{(}\AttributeTok{knitr.kable.NA =} \StringTok{\textquotesingle{}\textquotesingle{}}\NormalTok{)}

\NormalTok{packages }\OtherTok{\textless{}{-}} \FunctionTok{c}\NormalTok{(}\StringTok{\textquotesingle{}gridExtra\textquotesingle{}}\NormalTok{, }\StringTok{\textquotesingle{}BayesFactor\textquotesingle{}}\NormalTok{, }\StringTok{\textquotesingle{}tidyverse\textquotesingle{}}\NormalTok{, }\StringTok{"RColorBrewer"}\NormalTok{, }\StringTok{"lme4"}\NormalTok{, }\StringTok{"sjPlot"}\NormalTok{, }\StringTok{"lsr"}\NormalTok{, }\StringTok{"brms"}\NormalTok{, }\StringTok{"kableExtra"}\NormalTok{, }\StringTok{"afex"}\NormalTok{, }\StringTok{"emmeans"}\NormalTok{, }\StringTok{"viridis"}\NormalTok{, }\StringTok{"ggpubr"}\NormalTok{, }\StringTok{"hms"}\NormalTok{, }\StringTok{"scales"}\NormalTok{, }\StringTok{"cowplot"}\NormalTok{, }\StringTok{"gtsummary"}\NormalTok{, }\StringTok{"webshot"}\NormalTok{, }\StringTok{"webshot2"}\NormalTok{)}
\FunctionTok{lapply}\NormalTok{(packages, require, }\AttributeTok{character.only =} \ConstantTok{TRUE}\NormalTok{)}

\FunctionTok{set.seed}\NormalTok{(}\DecValTok{0815}\NormalTok{)}

\CommentTok{\# file with various statistical functions, among other things it provides tests for Bayes Factors (BFs)}
\FunctionTok{source}\NormalTok{(}\StringTok{\textquotesingle{}statisticalTests.R\textquotesingle{}}\NormalTok{)}

\CommentTok{\# Wrapper for brm models such that it saves the full model the first time it is run, otherwise it loads it from disk}
\NormalTok{run\_model }\OtherTok{\textless{}{-}} \ControlFlowTok{function}\NormalTok{(expr, modelName, }\AttributeTok{path=}\StringTok{\textquotesingle{}brm\textquotesingle{}}\NormalTok{, }\AttributeTok{reuse =} \ConstantTok{TRUE}\NormalTok{) \{}
\NormalTok{  path }\OtherTok{\textless{}{-}} \FunctionTok{paste0}\NormalTok{(path,}\StringTok{\textquotesingle{}/\textquotesingle{}}\NormalTok{, modelName, }\StringTok{".brm"}\NormalTok{)}
  \ControlFlowTok{if}\NormalTok{ (reuse) \{}
\NormalTok{    fit }\OtherTok{\textless{}{-}} \FunctionTok{suppressWarnings}\NormalTok{(}\FunctionTok{try}\NormalTok{(}\FunctionTok{readRDS}\NormalTok{(path), }\AttributeTok{silent =} \ConstantTok{TRUE}\NormalTok{))}
\NormalTok{  \}}
  \ControlFlowTok{if}\NormalTok{ (}\FunctionTok{is}\NormalTok{(fit, }\StringTok{"try{-}error"}\NormalTok{)) \{}
\NormalTok{    fit }\OtherTok{\textless{}{-}} \FunctionTok{eval}\NormalTok{(expr)}
    \FunctionTok{saveRDS}\NormalTok{(fit, }\AttributeTok{file =}\NormalTok{ path)}
\NormalTok{  \}}
\NormalTok{  fit}
\NormalTok{\}}

\CommentTok{\# Setting some plotting params}
\NormalTok{w\_box          }\OtherTok{\textless{}{-}} \FloatTok{0.2}      \CommentTok{\# width of boxplot, also used for jittering points and lines    }
\NormalTok{line\_jitter    }\OtherTok{\textless{}{-}}\NormalTok{ w\_box }\SpecialCharTok{/} \DecValTok{2}
\NormalTok{xAnnotate      }\OtherTok{\textless{}{-}} \SpecialCharTok{{-}}\FloatTok{0.3}

\CommentTok{\# jitter params}
\NormalTok{jit\_height  }\OtherTok{\textless{}{-}} \FloatTok{0.01}
\NormalTok{jit\_width   }\OtherTok{\textless{}{-}} \FloatTok{0.05}
\NormalTok{jit\_alpha   }\OtherTok{\textless{}{-}} \FloatTok{0.6}

\CommentTok{\# colors for age groups}
\NormalTok{groupcolors }\OtherTok{\textless{}{-}} \FunctionTok{c}\NormalTok{(}\StringTok{"\#1b9e77"}\NormalTok{, }\StringTok{"\#d95f02"}\NormalTok{, }\StringTok{"\#7570b3"}\NormalTok{)}
\end{Highlighting}
\end{Shaded}

\chapter{Abstract}\label{abstract}

We investigated how patients with Parkinson's disease (PD) balance the
explore-exploit trade-off using a spatially correlated bandit task,
where the spatial structure of rewards facilitated value generalization
(i.e., nearby options yield similar rewards). Participants were tested
either shortly after taking Levodopa (L-Dopa) or just before their next
scheduled dose. Patients with polyneuropathy served as a control group,
comparable in age, depressive symptoms, and basic cognitive functioning.

We conducted both behavioral and computational analyses, revealing
distinct behavioral patterns and computational signatures. PD patients
on L-Dopa effectively balanced exploration and exploitation to obtain
rewards, though not as efficiently as polyneuropathy patients. In stark
contrast, participants off L-Dopa rarely exploited known high-value
options and primarily explored novel ones. This overreliance on
exploration impaired their ability to navigate the explore-exploit
trade-off and maximize rewards.

To better understand the mechanisms underlying these behavioral
differences, we employed a computational approach using the Gaussian
Process Upper Confidence Bound (GP-UCB) model. This model integrates
similarity-based generalization with two distinct exploration
mechanisms: directed exploration, which seeks to reduce uncertainty
about rewards, and random exploration, which introduces stochastic
variability in choice behavior. Analysis of the model parameters showed
that behavioral differences between the on- and off-medication
conditions were primarily driven by differences in uncertainty-directed
exploration, while the level of random exploration remained unchanged.
Both PD groups exhibited lower generalization than the control group,
which may underlie their overall poorer performance.

In sum, PD patients off L-Dopa favored exploration over exploitation,
leading to impaired reward maximization, while those on medication
balanced the explore-exploit trade-off more successfully.

\chapter{Intro}\label{intro}

A central distinction between different forms of exploration behavior is
that \emph{directed exploration} reflects the drive for knowledge about
novel options, whereas \emph{undirected exploration} refers to random
variability in the choice process (Giron et al., 2023; Meder et al.,
2021; Sadeghiyeh et al., 2020; Schulz et al., 2019; Wu et al., 2018).

\chapter{Experiment}\label{experiment}

We investigated how patients with Parkinson's disease (PD) manage the
explore-exploit trade-off using a spatially correlated multi-armed
bandit task. Participants accumulated rewards by selecting tiles
(options) with normally distributed rewards. The spatial correlation
between rewards facilitated generalization, allowing participants to
adapt to the structure of the environment and balance exploring new
options versus exploiting known high-reward options.

\begin{figure}[H]

{\centering \pandocbounded{\includegraphics[keepaspectratio]{img/task_overview_Giron_et_al_2021.png}}

}

\caption{Screenshot from experiment and example environments (from Giron
et al., 2023)}

\end{figure}%

\section{Materials and procedure}\label{materials-and-procedure}

40 distinct environments were generated using a radial basis function
kernel with \(\lambda = 4\), creating a bivariate reward function on a
grid that maps each tile location to a specific reward value. These
smooth reward functions gradually varied across the grid.

Participants completed 10 rounds of the task, each featuring a new
environment drawn without replacement from the set of 40 environments.
In each round, participants had 25 choices to maximize rewards. The
first round served as a tutorial to familiarize participants with the
task and was excluded from the analyses. The final round (round 10) was
a bonus round where, after 15 choices, participants were asked to
predict rewards for five unrevealed options. Data from this round were
also excluded from the main analysis and analyzed separately.

At the start of each round, one tile was randomly revealed, and
participants sequentially sampled 25 tiles. On each trial, they could
choose to either click a new tile or re-click a previously selected
tile. Selections were made by selecting the tile on the computer screen,
upon which the received a reward arbitrarily scaled to the range
{[}0,50{]}. Re-clicked tiles showed small variations in reward due to
normally distributed noise.

\section{Sample}\label{sample}

We collected data from adult participants with Parkinson's disease (PD)
who receive Levodopa (L-Dopa) for treatment (Abbott, 2010; Tambasco et
al., 2018). L-Dopa is a pharmacological agent used to treat the symptoms
of Parkinson's disease, which is associated with a deficiency of
dopamine. Since dopamine cannot pass the blood-brain barrier, L-Dopa is
used as a precursor, which is absorbed through the gastrointestinal
tract and subsequently metabolized to dopamine.

PD patients were randomly assigned to two conditions: on medication
(PD+) and off medication (PD-). In the PD+ group (\emph{n}=22), L-Dopa
was administered at least 30 minutes before the start of the experiment
to allow the drug sufficient time to take effect during the task. In the
PD- group (\emph{n}=20), the next scheduled dose for participants was
timed such that they were in a low dopamine state during the experiment,
offering a clear contrast to the PD+ group. Thus, we refer to the `on
medication' condition as the state after taking L-Dopa and the `off
medication' condition as the state before their next scheduled dose.

Patients with polyneuropathies (PNP) served as control group. The PNP
group (\emph{n}=20) consisted of individuals with polyneuropathies,
which is associated with physical symptoms similar to the motor
impairments seen in Parkinson's disease. However, since polyneuropathy
primarily affects peripheral nerves it is typically not associated with
cognitive impairments, enabling a comparison in terms of physical
symptomatology and the resulting burden of suffering.

Possible participants with Parkinson's disease were evaluated based on
Hoehn-Yahr scores recorded in their patient files. The scale, which
ranges from 1 to 5, assesses disease severity and motor impairments,
with higher scores indicating greater severity (Goetz et al., 2004;
Hoehn \& Yahr, 1967). We limited recruitment to individuals with scores
between 1 and 3, as scores of 4 and 5 reflect severe impairment.

\section{Questionnaires}\label{questionnaires}

\subsection{Beck's depression inventory II
(BDI-II)}\label{becks-depression-inventory-ii-bdi-ii}

All participants answered the German version of the Beck Depression
Inventory II, a self-report inventory consisting of 21 items measuring
the severity of depression (Beck et al., 1996; Hautzinger et al., 2006).
Each item is scored from 0 to 3, with higher values representing more
severe symptoms, such that the total score ranges between 0 and 63. A
score of 0-11 is considered normal, 12-19 points indicate mild
depressive symptoms, while values of 20 or higher suggest moderate to
severe depression.

\begin{tcolorbox}[enhanced jigsaw, bottomtitle=1mm, leftrule=.75mm, colback=white, toptitle=1mm, breakable, title=\textcolor{quarto-callout-note-color}{\faInfo}\hspace{0.5em}{Note}, colbacktitle=quarto-callout-note-color!10!white, opacitybacktitle=0.6, coltitle=black, colframe=quarto-callout-note-color-frame, left=2mm, opacityback=0, bottomrule=.15mm, rightrule=.15mm, titlerule=0mm, arc=.35mm, toprule=.15mm]

TO DO: check BDI scoring Wikipedia: Die Nationale Versorgungsleitlinie
Unipolare Depression{[}3{]} listet im Anhang 1 S.217 (``Schwellenwerte
bei psychometrischen Testverfahren'') folgende Grenzwerte für das BDI-II
auf: https://www.leitlinien.de/themen/depression
https://doi.org/10.6101/AZQ\%2F000496

\begin{itemize}
\tightlist
\item
  0--13: keine Depression bzw. klinisch unauffällig oder remittiert
\item
  14--19: leichtes depressives Syndrom
\item
  20--28: mittelgradiges depressives Syndrom
\item
  ≥ 29: schweres depressives Syndrom
\end{itemize}

\end{tcolorbox}

\subsection{Mini-Mental State Examination
(MMSE)}\label{mini-mental-state-examination-mmse}

The Mini-Mental State Examination (MMSE) is a clinical test used assess
cognitive function and impairment, frequently used in in patients with
dementia (Folstein et al., 1975). The test comprises 30 questions
pertaining to different domains, including as memory (e.g., recalling
three objects), temporal and spatial orientation (e.g., date and
location), arithmetic ability, and various others. Each correct answer
receives one point, up to a maximum score of 30 points. In this study,
an MMSE score between 27 and 30 points was defined as a criterion to
establish a clear cutoff for excluding the presence of dementia or
significant cognitive impairment, but none of the patients received such
high scores.

\subsection{Hoehn-Yahr scale for severity of
Parkinson}\label{hoehn-yahr-scale-for-severity-of-parkinson}

The Hoehn-Yahr scale assesses disease severity and motor impairments
(Hoehn \& Yahr, 1967). Participants can receive a score between one and
five, with higher scores indicating more severe problems. We limited our
recruitment to patients with scores ranging from to 1 to 3.

\chapter{Behavioral data}\label{behavioral-data}

All behavioral data are stored in file
\emph{data\_gridsearch\_parkinson.csv}.

\begin{Shaded}
\begin{Highlighting}[]
\CommentTok{\# read in data}
\NormalTok{dat       }\OtherTok{\textless{}{-}} \FunctionTok{read\_delim}\NormalTok{(}\StringTok{"data/data\_gridsearch\_parkinson.csv"}\NormalTok{, }
                        \AttributeTok{delim =} \StringTok{","}\NormalTok{,}
                        \AttributeTok{col\_types =} \FunctionTok{cols}\NormalTok{(}
                          \CommentTok{\#id = readr::col\_factor(),}
                          \AttributeTok{group =}\NormalTok{ readr}\SpecialCharTok{::}\FunctionTok{col\_factor}\NormalTok{(),}
                          \AttributeTok{gender =}\NormalTok{ readr}\SpecialCharTok{::}\FunctionTok{col\_factor}\NormalTok{(),}
                          \AttributeTok{z =} \FunctionTok{col\_double}\NormalTok{(),        }
                          \AttributeTok{zscaled =} \FunctionTok{col\_double}\NormalTok{(),}
                          \AttributeTok{hoehn\_yahr =} \FunctionTok{col\_double}\NormalTok{()}
\NormalTok{                        )) }\CommentTok{\#\%\textgreater{}\% }
  \CommentTok{\# arrange(as.numeric(as.character(id)))  }

\CommentTok{\# clean up}
\NormalTok{dat }\OtherTok{\textless{}{-}}\NormalTok{ dat }\SpecialCharTok{\%\textgreater{}\%}
  \FunctionTok{select}\NormalTok{(}\SpecialCharTok{{-}}\NormalTok{condition, }\SpecialCharTok{{-}}\NormalTok{comments) }\SpecialCharTok{\%\textgreater{}\%} 
  \FunctionTok{mutate}\NormalTok{(}\AttributeTok{group =} \FunctionTok{case\_match}\NormalTok{(group,}
                            \StringTok{"PPD{-}"} \SpecialCharTok{\textasciitilde{}} \StringTok{"PD{-}"}\NormalTok{,}
                            \StringTok{"PPD+"} \SpecialCharTok{\textasciitilde{}} \StringTok{"PD+"}\NormalTok{,}
                            \StringTok{"PNP"} \SpecialCharTok{\textasciitilde{}} \StringTok{"PNP"}\NormalTok{,}
                            \AttributeTok{.default =} \ConstantTok{NA}\NormalTok{)) }\SpecialCharTok{\%\textgreater{}\%} 
  \FunctionTok{mutate}\NormalTok{(}\AttributeTok{group =} \FunctionTok{factor}\NormalTok{(group, }\AttributeTok{levels =} \FunctionTok{c}\NormalTok{(}\StringTok{"PNP"}\NormalTok{, }\StringTok{"PD+"}\NormalTok{, }\StringTok{"PD{-}"}\NormalTok{)))  }\SpecialCharTok{\%\textgreater{}\%} 
  \FunctionTok{mutate}\NormalTok{(}\AttributeTok{type\_choice =} \FunctionTok{factor}\NormalTok{(type\_choice, }\AttributeTok{levels =} \FunctionTok{c}\NormalTok{(}\StringTok{"Repeat"}\NormalTok{, }\StringTok{"Near"}\NormalTok{, }\StringTok{"Far"}\NormalTok{)))  }\SpecialCharTok{\%\textgreater{}\%} 
  \FunctionTok{mutate}\NormalTok{(}\AttributeTok{gender =} \FunctionTok{recode}\NormalTok{(gender, }\StringTok{"w"} \OtherTok{=} \StringTok{"f"}\NormalTok{)) }\SpecialCharTok{\%\textgreater{}\%} 
  \FunctionTok{rename}\NormalTok{(}\AttributeTok{MMSE =} \StringTok{\textasciigrave{}}\AttributeTok{mini\_mental}\StringTok{\textasciigrave{}}\NormalTok{) }\SpecialCharTok{\%\textgreater{}\%} 
  \FunctionTok{mutate}\NormalTok{(}\AttributeTok{last\_ldopa =} \FunctionTok{if\_else}\NormalTok{(group }\SpecialCharTok{!=} \StringTok{"PNP"}\NormalTok{, }\FunctionTok{as\_hms}\NormalTok{(last\_ldopa), }\FunctionTok{as\_hms}\NormalTok{(}\ConstantTok{NA}\NormalTok{)),}
         \AttributeTok{next\_ldopa =} \FunctionTok{if\_else}\NormalTok{(group }\SpecialCharTok{!=} \StringTok{"PNP"}\NormalTok{, }\FunctionTok{as\_hms}\NormalTok{(next\_ldopa), }\FunctionTok{as\_hms}\NormalTok{(}\ConstantTok{NA}\NormalTok{)),}
         \AttributeTok{time\_exp =} \FunctionTok{if\_else}\NormalTok{(group }\SpecialCharTok{!=} \StringTok{"PNP"}\NormalTok{, }\FunctionTok{as\_hms}\NormalTok{(time\_exp), }\FunctionTok{as\_hms}\NormalTok{(}\ConstantTok{NA}\NormalTok{))) }\SpecialCharTok{\%\textgreater{}\%} 
  \FunctionTok{mutate}\NormalTok{(}\AttributeTok{time\_since\_ldopa =} \FunctionTok{as.numeric}\NormalTok{(time\_exp }\SpecialCharTok{{-}}\NormalTok{ last\_ldopa, }\AttributeTok{unit =} \StringTok{"mins"}\NormalTok{))}

\CommentTok{\# get subject information}
\NormalTok{df\_sample }\OtherTok{\textless{}{-}}\NormalTok{ dat }\SpecialCharTok{\%\textgreater{}\%} 
  \FunctionTok{select}\NormalTok{(id, age, gender,group,BDI,MMSE,hoehn\_yahr,last\_ldopa,next\_ldopa,time\_exp,time\_since\_ldopa) }\SpecialCharTok{\%\textgreater{}\%} 
  \FunctionTok{group\_by}\NormalTok{(id) }\SpecialCharTok{\%\textgreater{}\%}
  \FunctionTok{slice\_head}\NormalTok{(}\AttributeTok{n =} \DecValTok{1}\NormalTok{) }\SpecialCharTok{\%\textgreater{}\%} 
  \FunctionTok{arrange}\NormalTok{(group)}

\ControlFlowTok{if}\NormalTok{ (knitr}\SpecialCharTok{::}\FunctionTok{is\_html\_output}\NormalTok{()) \{}
  \FunctionTok{head}\NormalTok{(dat) }\SpecialCharTok{\%\textgreater{}\%}
    \FunctionTok{kable}\NormalTok{(}\StringTok{"html"}\NormalTok{, }\AttributeTok{caption =} \StringTok{"Behavioral data."}\NormalTok{) }\SpecialCharTok{\%\textgreater{}\%}
    \FunctionTok{kable\_styling}\NormalTok{(}\AttributeTok{bootstrap\_options =} \FunctionTok{c}\NormalTok{(}\StringTok{"striped"}\NormalTok{, }\StringTok{"hover"}\NormalTok{, }\StringTok{"condensed"}\NormalTok{), }\AttributeTok{full\_width =} \ConstantTok{FALSE}\NormalTok{) }\SpecialCharTok{\%\textgreater{}\%} 
    \FunctionTok{scroll\_box}\NormalTok{(}\AttributeTok{width =} \StringTok{"100\%"}\NormalTok{, }\AttributeTok{height =} \StringTok{"300px"}\NormalTok{)}
\NormalTok{\} }\ControlFlowTok{else}\NormalTok{ \{}
  \FunctionTok{head}\NormalTok{(dat) }\SpecialCharTok{\%\textgreater{}\%}
    \FunctionTok{kable}\NormalTok{(}\StringTok{"latex"}\NormalTok{, }\AttributeTok{caption =} \StringTok{"Behavioral data."}\NormalTok{) }
\NormalTok{\}}
\end{Highlighting}
\end{Shaded}

\begin{table}

\caption{Behavioral data.}
\centering
\begin{tabular}[t]{r|r|r|r|r|r|r|r|r|r|r|l|r|r|l|l|r|r|r|l|l|l|l|r}
\hline
id & session & x & y & chosen & z & zscaled & time & trial & round & distance & type\_choice & previous\_reward & age & gender & group & BDI & MMSE & hoehn\_yahr & last\_ldopa & next\_ldopa & time\_exp & wearing\_off\_min & time\_since\_ldopa\\
\hline
115 & 1 & 4 & 4 & 37 & 20 & 18 & 1.72e+17 & 0 & 1 &  &  &  & 60 & f & PNP & 11 & 27 &  &  &  &  &  & \\
\hline
115 & 1 & 4 & 2 & 21 & 41 & 31 & 1.72e+17 & 1 & 1 & 2 & Far & 20 & 60 & f & PNP & 11 & 27 &  &  &  &  &  & \\
\hline
115 & 1 & 3 & 2 & 20 & 45 & 34 & 1.72e+17 & 2 & 1 & 1 & Near & 41 & 60 & f & PNP & 11 & 27 &  &  &  &  &  & \\
\hline
115 & 1 & 6 & 3 & 31 & 29 & 24 & 1.72e+17 & 3 & 1 & 4 & Far & 45 & 60 & f & PNP & 11 & 27 &  &  &  &  &  & \\
\hline
115 & 1 & 3 & 1 & 12 & 38 & 29 & 1.72e+17 & 4 & 1 & 5 & Far & 29 & 60 & f & PNP & 11 & 27 &  &  &  &  &  & \\
\hline
115 & 1 & 2 & 1 & 11 & 37 & 29 & 1.72e+17 & 5 & 1 & 1 & Near & 38 & 60 & f & PNP & 11 & 27 &  &  &  &  &  & \\
\hline
\end{tabular}
\end{table}

The data frame \emph{dat} with the raw data contains the following
variables:

\begin{itemize}
\tightlist
\item
  \emph{id}: participant id
\item
  \emph{age} is participant age in years
\item
  \emph{gender}: (m)ale, (f)emale, (d)iverse
\item
  \emph{x} and \emph{y} are the sampled coordinates on the grid
\item
  \emph{chosen}: are the \emph{x} and \emph{y} coordinates of the chosen
  tile
\item
  \emph{z} is the reward obtained from the chosen tile. Re-clicked tiles
  could show small variations in the observed color (i.e., underlying
  reward) due to normally distributednoise,\(\epsilon∼N(0,1)\).
\item
  \emph{z\_scaled} is the observed outcome (reward), scaled in each
  round to a randomly drawn maximum value in the range of 70\% to 90\%
  of the darkest reward value
\item
  \emph{trial} is the trial number (0-25), with 0 corresponding to the
  initially revealed random tile.
\item
  \emph{round} is the round number (1 through 10), with 1=practice round
  (not analyzed) and 10=bonus round (analyzed only for bonus round
  judgments)
\item
  \emph{distance} is the Manhattan distance between consecutive clicks.
  \emph{NA} for trial 0, i.e., initially revealed random tile.
\item
  \emph{type\_choice} categorizes consecutive clicks as ``repeat''
  (clicking the same tile as in the previous round), ``near'' (clicking
  a directly neighboring tile, i.e.~distance=1), and ``far'' (clicking a
  tile with distance \textgreater{} 1). \emph{NA} for trial 0, i.e., the
  initially revealed random tile.
\item
  \emph{previous\_reward} is the reward \emph{z} obtained on the
  previous step. \emph{NA} for trial 0, i.e., the initially revealed
  random tile.
\item
  \emph{last\_ldopa}: time of the last L-Dopa dose (HH:MM)
\item
  \emph{next\_ldopa}: scheduled time of the next L-Dopa dose (HH:MM)
\item
  \emph{time\_exp}: time of the experiment (HH:MM)
\item
  \emph{time\_since\_ldopa}: time since last L-Dopa (in minutes)
\item
  \emph{wearing\_off\_min}:
\end{itemize}

\section{Sample characteristics}\label{sample-characteristics}

We collected data from 62 adult participants with Parkinson disease, on
and off L-Dopa medication, and Polyneuropathy. Let's have a look at the
sample:

\begin{Shaded}
\begin{Highlighting}[]
\NormalTok{df\_sample }\SpecialCharTok{\%\textgreater{}\%} 
  \FunctionTok{group\_by}\NormalTok{(group) }\SpecialCharTok{\%\textgreater{}\%} 
  \FunctionTok{summarise}\NormalTok{(}\AttributeTok{n =} \FunctionTok{n}\NormalTok{(),}
            \AttributeTok{female =} \FunctionTok{sum}\NormalTok{(gender }\SpecialCharTok{==} \StringTok{"f"}\NormalTok{),}
            \AttributeTok{mean\_age =} \FunctionTok{mean}\NormalTok{(age),}
            \AttributeTok{sd\_age =} \FunctionTok{sd}\NormalTok{(age),}
            \AttributeTok{mean\_BDI =} \FunctionTok{mean}\NormalTok{(BDI, }\AttributeTok{na.rm=}\NormalTok{ T), }
            \AttributeTok{mean\_MMSE =} \FunctionTok{mean}\NormalTok{(MMSE, }\AttributeTok{na.rm=}\NormalTok{ T),}
            \AttributeTok{mean\_HY =} \FunctionTok{mean}\NormalTok{(hoehn\_yahr, }\AttributeTok{na.rm=}\NormalTok{ T),}
            \AttributeTok{mean\_time\_since\_ldopa =} \FunctionTok{mean}\NormalTok{(time\_since\_ldopa,  }\AttributeTok{na.rm=}\NormalTok{ T)) }\SpecialCharTok{\%\textgreater{}\%} 
\NormalTok{  \{}
    \ControlFlowTok{if}\NormalTok{ (knitr}\SpecialCharTok{::}\FunctionTok{is\_html\_output}\NormalTok{()) \{}
      \FunctionTok{kable}\NormalTok{(., }\AttributeTok{format =} \StringTok{"html"}\NormalTok{, }\AttributeTok{escape =} \ConstantTok{FALSE}\NormalTok{, }\AttributeTok{digits =} \DecValTok{1}\NormalTok{) }\SpecialCharTok{\%\textgreater{}\%}
        \FunctionTok{kable\_styling}\NormalTok{(}\StringTok{"striped"}\NormalTok{, }\AttributeTok{full\_width =} \ConstantTok{FALSE}\NormalTok{)}
\NormalTok{    \} }\ControlFlowTok{else}\NormalTok{ \{}
      \FunctionTok{kable}\NormalTok{(., }\AttributeTok{format =} \StringTok{"latex"}\NormalTok{, }\AttributeTok{digits =} \DecValTok{1}\NormalTok{) }
\NormalTok{    \}}
\NormalTok{  \}}
\end{Highlighting}
\end{Shaded}

\begin{tabular}{l|r|r|r|r|r|r|r|r}
\hline
group & n & female & mean\_age & sd\_age & mean\_BDI & mean\_MMSE & mean\_HY & mean\_time\_since\_ldopa\\
\hline
PNP & 20 & 13 & 63.9 & 8.5 & 8.1 & 29.1 &  & \\
\hline
PD+ & 22 & 11 & 61.6 & 5.8 & 8.0 & 29.2 & 1.9 & 107.0\\
\hline
PD- & 20 & 9 & 65.8 & 7.9 & 7.2 & 29.0 & 2.0 & 249.5\\
\hline
\end{tabular}

We analyzed the behavioral data in terms of performance and exploration
behavior. These analyses exclude the tutorial and bonus rounds, leaving
a total of 200 search decisions (8 rounds \(\times\) 25 trials) for each
participant. We then report the results of the bonus round, where we
analyze participants' reward predictions and confidence judgments. We
report both frequentist statistics and Bayes factors (\(BF\)) to
quantify the relative evidence of the data in favor of the alternative
hypothesis (\(H_A\)) over the null hypothesis (\(H_0\)); see Appendix
for details and references. Various helper functions are implemented in
\emph{statisticalTests.R}. Regression analyses were performed in a
Bayesian framework with \emph{Stan}, accessed via R-package \emph{brms},
complemented by frequentist hierachical regression analyses (via package
\emph{lmer}).

\section{Performance: Rewards by
round}\label{performance-rewards-by-round}

First, let's plot the obtained rewards as function of group and round.

\begin{center}
\pandocbounded{\includegraphics[keepaspectratio]{gridsearch_parkinson_behavioral_analyses_files/figure-pdf/unnamed-chunk-4-1.pdf}}
\end{center}

ANOVA results with \emph{round} as within- and \emph{group} as
between-subjects factor show a difference between groups, with PNP
patients achieving the greatest rewards, followed by PD+ and PD-
patients, but no change across rounds (and no interaction). A Bayesian
regression analysis yield the same results. Therefore, for the
subsequent analyses we collapse across rounds.

\begin{Shaded}
\begin{Highlighting}[]
\NormalTok{aov\_rounds }\OtherTok{\textless{}{-}} \FunctionTok{aov\_ez}\NormalTok{(}
  \AttributeTok{id =} \StringTok{"id"}\NormalTok{,                 }
  \AttributeTok{dv =} \StringTok{"mean\_reward"}\NormalTok{,        }
  \AttributeTok{within =} \StringTok{"round"}\NormalTok{,          }
  \AttributeTok{between =} \StringTok{"group"}\NormalTok{,         }
  \AttributeTok{data =}\NormalTok{ df\_mean\_reward\_subject\_by\_round}
\NormalTok{)}

\ControlFlowTok{if}\NormalTok{ (knitr}\SpecialCharTok{::}\FunctionTok{is\_html\_output}\NormalTok{()) \{}
  \FunctionTok{kable}\NormalTok{(}\FunctionTok{as.data.frame}\NormalTok{(aov\_rounds}\SpecialCharTok{$}\NormalTok{anova\_table), }
        \AttributeTok{format =} \StringTok{"html"}\NormalTok{, }\AttributeTok{escape =} \ConstantTok{FALSE}\NormalTok{, }\AttributeTok{digits =} \DecValTok{2}\NormalTok{, }
        \AttributeTok{caption =} \StringTok{"ANOVA results with round as within{-}subjects factor and group as between subjects factor, where rewards per round were first aggregated within subjects."}\NormalTok{) }\SpecialCharTok{\%\textgreater{}\%}
    \FunctionTok{kable\_styling}\NormalTok{(}\StringTok{"striped"}\NormalTok{, }\AttributeTok{full\_width =} \ConstantTok{FALSE}\NormalTok{)}
\NormalTok{\} }\ControlFlowTok{else}\NormalTok{ \{}
  \FunctionTok{kable}\NormalTok{(}\FunctionTok{as.data.frame}\NormalTok{(aov\_rounds}\SpecialCharTok{$}\NormalTok{anova\_table), }
        \AttributeTok{format =} \StringTok{"latex"}\NormalTok{, }\AttributeTok{digits =} \DecValTok{2}\NormalTok{, }
        \AttributeTok{caption =} \StringTok{"ANOVA results with round as within{-}subjects factor and group as between subjects factor, where rewards per round were first aggregated within subjects."}\NormalTok{)}
\NormalTok{\}}
\end{Highlighting}
\end{Shaded}

\begin{table}

\caption{ANOVA results with round as within-subjects factor and group as between subjects factor, where rewards per round were first aggregated within subjects.}
\centering
\begin{tabular}[t]{l|r|r|r|r|r|r}
\hline
  & num Df & den Df & MSE & F & ges & Pr(>F)\\
\hline
group & 2.00 & 59.0 & 86.59 & 21.17 & 0.17 & 0.00\\
\hline
round & 6.10 & 359.7 & 34.93 & 1.31 & 0.02 & 0.25\\
\hline
group:round & 12.19 & 359.7 & 34.93 & 1.30 & 0.03 & 0.22\\
\hline
\end{tabular}
\end{table}

\begin{Shaded}
\begin{Highlighting}[]
\NormalTok{brm\_rounds }\OtherTok{\textless{}{-}} \FunctionTok{run\_model}\NormalTok{(}\FunctionTok{brm}\NormalTok{(}
\NormalTok{  mean\_reward }\SpecialCharTok{\textasciitilde{}}\NormalTok{ round }\SpecialCharTok{*}\NormalTok{ group }\SpecialCharTok{+}\NormalTok{ (}\DecValTok{1} \SpecialCharTok{|}\NormalTok{ id),   }\CommentTok{\# Random intercept for subject}
  \AttributeTok{data =}\NormalTok{ df\_mean\_reward\_subject\_by\_round,   }
  \AttributeTok{family =} \FunctionTok{gaussian}\NormalTok{(),                      }
  \AttributeTok{iter =} \DecValTok{4000}\NormalTok{,                              }
  \AttributeTok{warmup =} \DecValTok{1000}\NormalTok{,                            }
  \AttributeTok{chains =} \DecValTok{4}\NormalTok{,                               }
  \AttributeTok{cores =} \DecValTok{4}\NormalTok{,                                }
  \AttributeTok{seed =} \DecValTok{0511}\NormalTok{),}
  \AttributeTok{modelName =} \StringTok{\textquotesingle{}brm\_reward\_rounds\textquotesingle{}}\NormalTok{)}

\CommentTok{\# Extract fitted values and add to data df}
\NormalTok{fitted\_values }\OtherTok{\textless{}{-}} \FunctionTok{fitted}\NormalTok{(brm\_rounds, }\AttributeTok{re\_formula =} \ConstantTok{NA}\NormalTok{)}
\NormalTok{df\_mean\_reward\_subject\_by\_round}\SpecialCharTok{$}\NormalTok{fitted\_mean\_reward }\OtherTok{\textless{}{-}}\NormalTok{ fitted\_values[, }\StringTok{"Estimate"}\NormalTok{]}

\FunctionTok{ggplot}\NormalTok{(df\_mean\_reward\_subject\_by\_round, }\FunctionTok{aes}\NormalTok{(}\AttributeTok{x =}\NormalTok{ round, }\AttributeTok{y =}\NormalTok{ mean\_reward, }\AttributeTok{group =}\NormalTok{ group, }\AttributeTok{shape =}\NormalTok{ group, }\AttributeTok{color =}\NormalTok{ group)) }\SpecialCharTok{+}
  \FunctionTok{geom\_point}\NormalTok{(}\AttributeTok{data =}\NormalTok{ df\_summary\_by\_round, }\FunctionTok{aes}\NormalTok{(}\AttributeTok{x =}\NormalTok{ round, }\AttributeTok{y =}\NormalTok{ mean\_of\_means, }\AttributeTok{shape =}\NormalTok{ group), }\AttributeTok{size =} \DecValTok{3}\NormalTok{) }\SpecialCharTok{+}
  \FunctionTok{geom\_line}\NormalTok{(}\FunctionTok{aes}\NormalTok{(}\AttributeTok{y =}\NormalTok{ fitted\_mean\_reward), }\AttributeTok{size =} \DecValTok{1}\NormalTok{) }\SpecialCharTok{+}  
  \FunctionTok{geom\_jitter}\NormalTok{(}\FunctionTok{aes}\NormalTok{(}\AttributeTok{x =}\NormalTok{ round, }\AttributeTok{y =}\NormalTok{ mean\_reward), }\AttributeTok{size =} \DecValTok{1}\NormalTok{, }\AttributeTok{alpha =} \FloatTok{0.3}\NormalTok{, }\AttributeTok{width =} \FloatTok{0.2}\NormalTok{) }\SpecialCharTok{+}
  \FunctionTok{scale\_y\_continuous}\NormalTok{(}\StringTok{"Mean Reward"}\NormalTok{, }\AttributeTok{breaks =} \FunctionTok{c}\NormalTok{(}\DecValTok{25}\NormalTok{,}\DecValTok{30}\NormalTok{,}\DecValTok{35}\NormalTok{)) }\SpecialCharTok{+}
  \FunctionTok{xlab}\NormalTok{(}\StringTok{"Round"}\NormalTok{) }\SpecialCharTok{+}
  \FunctionTok{scale\_fill\_manual}\NormalTok{(}\AttributeTok{values =}\NormalTok{ groupcolors) }\SpecialCharTok{+}
  \FunctionTok{scale\_color\_manual}\NormalTok{(}\AttributeTok{values =}\NormalTok{ groupcolors) }\SpecialCharTok{+}
  \FunctionTok{ggtitle}\NormalTok{(}\StringTok{"Mean Reward by Rounds and Group (brms)"}\NormalTok{) }\SpecialCharTok{+}
  \FunctionTok{theme\_classic}\NormalTok{() }\SpecialCharTok{+}
  \FunctionTok{theme}\NormalTok{(}\AttributeTok{legend.title =} \FunctionTok{element\_blank}\NormalTok{())}
\end{Highlighting}
\end{Shaded}

\begin{center}
\pandocbounded{\includegraphics[keepaspectratio]{gridsearch_parkinson_behavioral_analyses_files/figure-pdf/unnamed-chunk-5-1.pdf}}
\end{center}

\begin{Shaded}
\begin{Highlighting}[]
\CommentTok{\# tbl\_regression(brm\_rounds, exponentiate = F) }

\FunctionTok{tab\_model}\NormalTok{(brm\_rounds)}
\end{Highlighting}
\end{Shaded}

\begin{Shaded}
\begin{Highlighting}[]
\FunctionTok{plot\_model}\NormalTok{(brm\_rounds, }\AttributeTok{type =} \StringTok{"est"}\NormalTok{)}
\end{Highlighting}
\end{Shaded}

\begin{center}
\pandocbounded{\includegraphics[keepaspectratio]{gridsearch_parkinson_behavioral_analyses_files/figure-pdf/unnamed-chunk-5-2.pdf}}
\end{center}

\section{Performance: Rewards by
group}\label{performance-rewards-by-group}

The plot shows the overall performance of each group, where we compute
for each subjects the mean reward across all trials. Each dot is one
participants' mean reward across all rounds and trials.

\begin{tabular}{lrrrrrrrr}
\toprule
group & n & m\_reward & md\_reward & var\_reward & sd\_reward & se\_reward & lower\_ci\_reward & upper\_ci\_reward\\
\midrule
PNP & 20 & 34.04 & 34.20 & 17.97 & 4.24 & 0.95 & 32.06 & 36.03\\
PD+ & 22 & 31.45 & 31.63 & 7.36 & 2.71 & 0.58 & 30.24 & 32.65\\
PD- & 20 & 27.33 & 27.61 & 7.50 & 2.74 & 0.61 & 26.05 & 28.62\\
\bottomrule
\end{tabular}

\begin{center}
\pandocbounded{\includegraphics[keepaspectratio]{gridsearch_parkinson_behavioral_analyses_files/figure-pdf/unnamed-chunk-6-1.pdf}}
\end{center}

All groups differed, with PNP achieving the greatest amount of reward,
followed by PD+ and PD-.

\begin{itemize}
\tightlist
\item
  PNP vs.~PD+: \(t(40)=2.4\), \(p=.022\), \(d=0.7\), \(BF=2.7\)
\item
  PNP vs.~PD-: \(t(38)=5.9\), \(p<.001\), \(d=1.9\), \(BF>100\)
\item
  PD+ vs.~PD-: \(t(40)=2.4\), \(p=.022\), \(d=0.7\), \(BF=2.7\)
\end{itemize}

\section{Performance: Learning
curves}\label{performance-learning-curves}

Participants' learning curves illustrate the average reward obtained in
each trial across rounds. For both polyneuropathy patients (PNP) and
Parkinson's patients on medication (PD+), the mean rewards increased as
the round progresses, suggesting they effectively balanced exploration
and exploitation to maximize rewards. In stark contrast, Parkinson's
patients off medication (PD-) showed no improvement across trials.

TO DO: Add random reward as baseline

\begin{center}
\pandocbounded{\includegraphics[keepaspectratio]{gridsearch_parkinson_behavioral_analyses_files/figure-pdf/unnamed-chunk-8-1.pdf}}
\end{center}

\subsection{Performance: Role of physiological and cognitive assessments
(BDI, MMSE,
Hoehn-Yahr)}\label{performance-role-of-physiological-and-cognitive-assessments-bdi-mmse-hoehn-yahr}

We also assessed patients in terms of their depressive symptoms (via
BDI-II), cognitive functioning (via Mini-Mental-Status Examination,
MMSE), and severity of motor symptoms (via Hoehn-Yahr scale, Parkinson's
disease patients only).

We ran a hierarchical regression with reward as dependent variable and
group, BDI score, and MMSE score; with random intercepts for
participants to account for individual differences. This analysis
yielded only an effect of group, suggesting that BDI amd MMSE score were
not related to performance.

\begin{Shaded}
\begin{Highlighting}[]
\CommentTok{\# Hierarchical frequentist regression with random intercept: Reward as function of BDI and MMSE score (all patients)    }
\NormalTok{lmer\_performance\_BDI\_MMSE }\OtherTok{\textless{}{-}} \FunctionTok{lmer}\NormalTok{(z }\SpecialCharTok{\textasciitilde{}}\NormalTok{ group }\SpecialCharTok{+}\NormalTok{ BDI }\SpecialCharTok{+}\NormalTok{ MMSE }\SpecialCharTok{+}\NormalTok{ (}\DecValTok{1} \SpecialCharTok{|}\NormalTok{ id), }
                             \AttributeTok{data =} \FunctionTok{subset}\NormalTok{(dat, trial }\SpecialCharTok{\textgreater{}} \DecValTok{0} \SpecialCharTok{\&}\NormalTok{ round }\SpecialCharTok{\%in\%} \DecValTok{2}\SpecialCharTok{:}\DecValTok{9}\NormalTok{))}

\CommentTok{\#summary(lmer\_reward\_BDI\_MMSE)}

\FunctionTok{tab\_model}\NormalTok{(lmer\_performance\_BDI\_MMSE, }\AttributeTok{title =} \StringTok{"Hierarchical regression results: Performance as function of BDI and MMSE score."}\NormalTok{, }\AttributeTok{bpe=}\StringTok{"mean"}\NormalTok{)}
\end{Highlighting}
\end{Shaded}

\begin{Shaded}
\begin{Highlighting}[]
\CommentTok{\# Hierarchical Bayesian regression with random intercept: Reward as function of BDI and MMSE score (all patients)                              }
\NormalTok{brm\_performance\_BDI\_MMSE }\OtherTok{\textless{}{-}} \FunctionTok{run\_model}\NormalTok{(}\FunctionTok{brm}\NormalTok{(z }\SpecialCharTok{\textasciitilde{}}\NormalTok{ group }\SpecialCharTok{+}\NormalTok{ BDI }\SpecialCharTok{+}\NormalTok{ MMSE }\SpecialCharTok{+}\NormalTok{ (}\DecValTok{1}\SpecialCharTok{|}\NormalTok{id), }
                                       \AttributeTok{data=}\FunctionTok{subset}\NormalTok{(dat, trial }\SpecialCharTok{\textgreater{}} \DecValTok{0} \SpecialCharTok{\&}\NormalTok{ round }\SpecialCharTok{\%in\%} \DecValTok{2}\SpecialCharTok{:}\DecValTok{9}\NormalTok{ ), }
                                       \AttributeTok{cores=}\DecValTok{4}\NormalTok{,  }
                                       \AttributeTok{seed =} \DecValTok{0815}\NormalTok{,}
                                       \AttributeTok{iter =} \DecValTok{5000}\NormalTok{,}
                                       \AttributeTok{warmup=}\DecValTok{1000}\NormalTok{,}
                                       \AttributeTok{control =} \FunctionTok{list}\NormalTok{(}\AttributeTok{adapt\_delta =} \FloatTok{0.99}\NormalTok{, }\AttributeTok{max\_treedepth =} \DecValTok{15}\NormalTok{)),}
                                   \CommentTok{\#prior = prior(normal(0,10), class = "b")), }
                                   \AttributeTok{modelName =} \StringTok{\textquotesingle{}brm\_performance\_BDI\_MMSE\textquotesingle{}}\NormalTok{)}
\CommentTok{\#tab\_model(brm\_performance\_assessment, bpe="mean", title = "Hierarchical Bayesian regression: Performance as function of BDI and MMSE score.") }
\CommentTok{\#bayes\_R2(brm\_performance\_assessment) }
\CommentTok{\#tab\_model(lmer\_performance\_BDI\_MMSE, brm\_performance\_BDI\_MMSE, title = "Hierarchical regression results: Performance as function of BDI and MMSE score.", bpe="mean")}
\end{Highlighting}
\end{Shaded}

Next, we ran a hierarchical regression for Parkinson's patients only,
with reward as dependent variable and group, BDI, MMSE, and Hoehn-Yahr
score as predictors; with random intercepts for participants to account
for individual differences. This analysis only yielded an influence of
group, i.e.~being on or off L-Dopa.

\begin{Shaded}
\begin{Highlighting}[]
\CommentTok{\# Hierarchical frequentist regression with random intercept: Reward as function of BDI, MMSE, and Hoehner{-}Yahr score (Parkinson\textquotesingle{}s patients only)   }
\NormalTok{lmer\_reward\_PD\_only\_BDI\_MMSE\_HY }\OtherTok{\textless{}{-}} \FunctionTok{lmer}\NormalTok{(z }\SpecialCharTok{\textasciitilde{}}\NormalTok{ group }\SpecialCharTok{+}\NormalTok{ BDI }\SpecialCharTok{+}\NormalTok{ MMSE }\SpecialCharTok{+}\NormalTok{ hoehn\_yahr }\SpecialCharTok{+}\NormalTok{ (}\DecValTok{1} \SpecialCharTok{|}\NormalTok{ id), }
                                        \AttributeTok{data =} \FunctionTok{subset}\NormalTok{(dat, trial }\SpecialCharTok{\textgreater{}} \DecValTok{0} \SpecialCharTok{\&}\NormalTok{ round }\SpecialCharTok{\%in\%} \DecValTok{2}\SpecialCharTok{:}\DecValTok{9} \SpecialCharTok{\&}\NormalTok{ group }\SpecialCharTok{!=} \StringTok{"PNP"}\NormalTok{))}

\CommentTok{\#summary(lmer\_reward\_PD\_only\_BDI\_MMSE\_HY)}

\FunctionTok{tab\_model}\NormalTok{(lmer\_reward\_PD\_only\_BDI\_MMSE\_HY, }\AttributeTok{title =} \StringTok{"Hierarchical regression results: Performance of patients with Parkinson\textquotesingle{}s disease as function of BDI, MMSE, and Hoehn{-}Yahr score."}\NormalTok{,  }\AttributeTok{bpe=}\StringTok{"mean"}\NormalTok{)}
\end{Highlighting}
\end{Shaded}

\begin{Shaded}
\begin{Highlighting}[]
\CommentTok{\# Hierarchical Bayesian regression with random intercept: Reward as function of BDI, MMSE, and Hoehner{-}Yahr score (Parkinson\textquotesingle{}s patients only)                          }
\NormalTok{brm\_performance\_PD\_only\_BDI\_MMSE\_HY }\OtherTok{\textless{}{-}} \FunctionTok{run\_model}\NormalTok{(}\FunctionTok{brm}\NormalTok{(z }\SpecialCharTok{\textasciitilde{}}\NormalTok{ group }\SpecialCharTok{+}\NormalTok{ BDI }\SpecialCharTok{+}\NormalTok{ MMSE }\SpecialCharTok{+}\NormalTok{ (}\DecValTok{1}\SpecialCharTok{|}\NormalTok{id), }
                                       \AttributeTok{data=}\FunctionTok{subset}\NormalTok{(dat, trial }\SpecialCharTok{\textgreater{}} \DecValTok{0} \SpecialCharTok{\&}\NormalTok{ round }\SpecialCharTok{\%in\%} \DecValTok{2}\SpecialCharTok{:}\DecValTok{9} \SpecialCharTok{\&}\NormalTok{ group }\SpecialCharTok{!=} \StringTok{"PNP"}\NormalTok{), }
                                       \AttributeTok{cores=}\DecValTok{4}\NormalTok{, }
                                       \AttributeTok{seed =} \DecValTok{0815}\NormalTok{,}
                                       \AttributeTok{iter =} \DecValTok{5000}\NormalTok{,}
                                       \AttributeTok{warmup=}\DecValTok{1000}\NormalTok{,}
                                       \AttributeTok{control =} \FunctionTok{list}\NormalTok{(}\AttributeTok{adapt\_delta =} \FloatTok{0.99}\NormalTok{)),}
                                   \CommentTok{\#prior = prior(normal(0,10), class = "b")), }
                                   \AttributeTok{modelName =} \StringTok{\textquotesingle{}brm\_performance\_PD\_only\_BDI\_MMSE\_HY\textquotesingle{}}\NormalTok{)}

\CommentTok{\# tab\_model(lmer\_reward\_PD\_only\_BDI\_MMSE\_HY, brm\_performance\_PD\_only\_BDI\_MMSE\_HYtitle = "Hierarchical regression results: Performance of patients with Parkinson\textquotesingle{}s disease as function of BDI, MMSE, and Hoehn{-}Yahr score.",  bpe="mean")}
\end{Highlighting}
\end{Shaded}

\section{Exploration vs.~exploitation
choices}\label{exploration-vs.-exploitation-choices}

As a global measure of exploration behavior, we calculated the
proportion of unique tiles (out of 25) selected during each round.
Higher proportions indicate a greater tendency to explore novel options,
while lower proportions suggest a preference for exploiting known
options There were substantial differences between groups, with PD+
patients showing higher levels of exploratory behavior and, conversely,
a reduced tendency to exploit known options. This effect was
particularly pronounced for PD- patients, who almost exclusively
selected novels options during the task.

\begin{Shaded}
\begin{Highlighting}[]
\CommentTok{\# proportion of unique choices per round per subject}
\NormalTok{df\_unique\_choices\_round }\OtherTok{\textless{}{-}} 
\NormalTok{  dat }\SpecialCharTok{\%\textgreater{}\%}
  \FunctionTok{filter}\NormalTok{(round }\SpecialCharTok{\%in\%} \DecValTok{2}\SpecialCharTok{:}\DecValTok{9} \SpecialCharTok{\&}\NormalTok{ trial }\SpecialCharTok{\textgreater{}} \DecValTok{0}\NormalTok{) }\SpecialCharTok{\%\textgreater{}\%} 
  \FunctionTok{group\_by}\NormalTok{(id,group, round) }\SpecialCharTok{\%\textgreater{}\%}  
  \FunctionTok{summarize}\NormalTok{(}
    \AttributeTok{total =} \FunctionTok{n}\NormalTok{(),  }\CommentTok{\#  number of trials}
    \AttributeTok{unique\_tiles =} \FunctionTok{n\_distinct}\NormalTok{(x, y),  }\CommentTok{\# unique (x, y) combinations (i.e., tiles)}
    \AttributeTok{repeat\_tiles =}\NormalTok{ total }\SpecialCharTok{{-}}\NormalTok{ unique\_tiles}
\NormalTok{  ) }\SpecialCharTok{\%\textgreater{}\%} 
  \FunctionTok{mutate}\NormalTok{(}\AttributeTok{prop\_unique =}\NormalTok{ unique\_tiles}\SpecialCharTok{/}\NormalTok{total,}
         \AttributeTok{prop\_repeat =}\NormalTok{ repeat\_tiles}\SpecialCharTok{/}\NormalTok{total) }

\CommentTok{\# proportion of unique choices across 8 rounds per subject}
\NormalTok{df\_unique\_choices\_subject }\OtherTok{\textless{}{-}}\NormalTok{ df\_unique\_choices\_round }\SpecialCharTok{\%\textgreater{}\%} 
  \FunctionTok{group\_by}\NormalTok{(id, group) }\SpecialCharTok{\%\textgreater{}\%} 
  \FunctionTok{summarize}\NormalTok{(}\AttributeTok{m\_prop\_unique =} \FunctionTok{mean}\NormalTok{(prop\_unique),}
            \AttributeTok{m\_prop\_repeat =} \FunctionTok{mean}\NormalTok{(prop\_repeat))}


\FunctionTok{ggboxplot}\NormalTok{(df\_unique\_choices\_subject, }
          \AttributeTok{x =} \StringTok{"group"}\NormalTok{, }
          \AttributeTok{y =} \StringTok{"m\_prop\_unique"}\NormalTok{,}
          \AttributeTok{color =} \StringTok{"group"}\NormalTok{, }\AttributeTok{palette =}\NormalTok{groupcolors, }\AttributeTok{fill =} \StringTok{"group"}\NormalTok{, }\AttributeTok{alpha =} \FloatTok{0.2}\NormalTok{,}
          \AttributeTok{add =} \StringTok{"jitter"}\NormalTok{, }\AttributeTok{jitter.size =} \FloatTok{0.5}\NormalTok{, }\AttributeTok{shape =} \StringTok{"group"}\NormalTok{, }\AttributeTok{title =} \StringTok{"Proportion unique choices"}\NormalTok{) }\SpecialCharTok{+}
  \FunctionTok{scale\_y\_continuous}\NormalTok{(}\StringTok{"Mean proportion of unique tiles"}\NormalTok{, }
                     \AttributeTok{breaks =} \FunctionTok{c}\NormalTok{(}\DecValTok{0}\NormalTok{, }\FloatTok{0.25}\NormalTok{, }\FloatTok{0.5}\NormalTok{, }\FloatTok{0.75}\NormalTok{, }\DecValTok{1}\NormalTok{),  }
                     \AttributeTok{labels =} \FunctionTok{percent\_format}\NormalTok{(}\AttributeTok{accuracy =} \DecValTok{1}\NormalTok{)) }\SpecialCharTok{+}
  \FunctionTok{coord\_cartesian}\NormalTok{(}\AttributeTok{ylim=}\FunctionTok{c}\NormalTok{(}\DecValTok{0}\NormalTok{,}\FloatTok{1.25}\NormalTok{)) }\SpecialCharTok{+}
  \FunctionTok{xlab}\NormalTok{(}\StringTok{""}\NormalTok{) }\SpecialCharTok{+}
  \FunctionTok{stat\_compare\_means}\NormalTok{(}\AttributeTok{comparisons =} \FunctionTok{list}\NormalTok{( }\FunctionTok{c}\NormalTok{(}\StringTok{"PNP"}\NormalTok{, }\StringTok{"PD+"}\NormalTok{), }\FunctionTok{c}\NormalTok{(}\StringTok{"PD+"}\NormalTok{, }\StringTok{"PD{-}"}\NormalTok{)  ),}
                     \AttributeTok{paired =}\NormalTok{ F, }
                     \AttributeTok{method =} \StringTok{"t.test"}\NormalTok{, }
                     \CommentTok{\# label = "p.format",}
                     \FunctionTok{aes}\NormalTok{(}\AttributeTok{label =} \FunctionTok{paste0}\NormalTok{(}\StringTok{"p = "}\NormalTok{, }\FunctionTok{after\_stat}\NormalTok{(p.format)))}
\NormalTok{  ) }\SpecialCharTok{+}
  \FunctionTok{stat\_summary}\NormalTok{(}\AttributeTok{fun =}\NormalTok{ mean, }\AttributeTok{geom=}\StringTok{"point"}\NormalTok{, }\AttributeTok{shape =} \DecValTok{23}\NormalTok{, }\AttributeTok{fill =} \StringTok{"white"}\NormalTok{, }\AttributeTok{size=}\DecValTok{2}\NormalTok{) }\SpecialCharTok{+}
  \FunctionTok{theme\_classic}\NormalTok{() }\SpecialCharTok{+}
  \FunctionTok{theme}\NormalTok{(}\AttributeTok{strip.background =} \FunctionTok{element\_blank}\NormalTok{(),  }
        \AttributeTok{strip.text =} \FunctionTok{element\_text}\NormalTok{(}\AttributeTok{color =} \StringTok{"black"}\NormalTok{, }\AttributeTok{size=}\DecValTok{12}\NormalTok{),}
        \AttributeTok{legend.position =} \StringTok{"none"}
\NormalTok{  )  }
\end{Highlighting}
\end{Shaded}

\begin{center}
\pandocbounded{\includegraphics[keepaspectratio]{gridsearch_parkinson_behavioral_analyses_files/figure-pdf/unnamed-chunk-12-1.pdf}}
\end{center}

\begin{Shaded}
\begin{Highlighting}[]
\FunctionTok{ggsave}\NormalTok{(}\StringTok{"plots/unique\_choices.png"}\NormalTok{, }\AttributeTok{dpi=}\DecValTok{300}\NormalTok{, }\AttributeTok{width =}\DecValTok{5}\NormalTok{, }\AttributeTok{height=}\FloatTok{4.5}\NormalTok{)  }
\end{Highlighting}
\end{Shaded}

To investigate the temporal dynamics of exploration and exploitation, we
determined for each trial whether the chosen tile was novel (an
exploration decision) or had already been selected previously (an
exploitation decision). The data reveal that the likelihood of
exploiting a previously chosen option increased significantly over time
for both PNP and PD+ patients, indicating a gradual shift in behavior
from exploration to exploitation as the round progressed. In contrast,
PD- patients predominantly engaged in exploration throughout the round
and showed only a weak tendency towards increased exploitation as the
search horizon approached its end.

\begin{Shaded}
\begin{Highlighting}[]
\NormalTok{dat }\OtherTok{\textless{}{-}}\NormalTok{ dat }\SpecialCharTok{\%\textgreater{}\%}
  \FunctionTok{group\_by}\NormalTok{(id, round) }\SpecialCharTok{\%\textgreater{}\%}
  \FunctionTok{arrange}\NormalTok{(trial, }\AttributeTok{.by\_group =} \ConstantTok{TRUE}\NormalTok{) }\SpecialCharTok{\%\textgreater{}\%}  \CommentTok{\# Ensure data is sorted by trial}
  \FunctionTok{mutate}\NormalTok{(}
    \AttributeTok{is\_new =} \FunctionTok{factor}\NormalTok{(}\FunctionTok{if\_else}\NormalTok{(}\SpecialCharTok{!}\FunctionTok{duplicated}\NormalTok{(chosen), }\StringTok{"new"}\NormalTok{, }\StringTok{"repeat"}\NormalTok{))  }\CommentTok{\# Check uniqueness based on \textquotesingle{}chosen\textquotesingle{} column}
\NormalTok{  ) }\SpecialCharTok{\%\textgreater{}\%}
  \FunctionTok{ungroup}\NormalTok{()}

\NormalTok{dat\_repeat\_prop }\OtherTok{\textless{}{-}}\NormalTok{ dat }\SpecialCharTok{\%\textgreater{}\%}
  \FunctionTok{filter}\NormalTok{(trial }\SpecialCharTok{\textgreater{}} \DecValTok{0} \SpecialCharTok{\&}\NormalTok{ round }\SpecialCharTok{\%in\%} \DecValTok{2}\SpecialCharTok{:}\DecValTok{9}\NormalTok{) }\SpecialCharTok{\%\textgreater{}\%} 
  \FunctionTok{group\_by}\NormalTok{(id, group, trial) }\SpecialCharTok{\%\textgreater{}\%}
  \FunctionTok{summarize}\NormalTok{(}
    \AttributeTok{prop\_repeat =} \FunctionTok{mean}\NormalTok{(is\_new }\SpecialCharTok{==} \StringTok{"repeat"}\NormalTok{, }\AttributeTok{na.rm =} \ConstantTok{TRUE}\NormalTok{)  }\CommentTok{\# Calculate proportion of "repeat" choices}
\NormalTok{  )}

\CommentTok{\# plot proportion of repeat choices across trials}
\FunctionTok{ggplot}\NormalTok{(dat\_repeat\_prop, }\FunctionTok{aes}\NormalTok{(}\AttributeTok{x =}\NormalTok{ trial, }\AttributeTok{y =}\NormalTok{ prop\_repeat, }\AttributeTok{fill =}\NormalTok{ group, }\AttributeTok{shape =}\NormalTok{ group, }\AttributeTok{color =}\NormalTok{ group)) }\SpecialCharTok{+}
  \FunctionTok{stat\_summary}\NormalTok{(}\AttributeTok{fun.data =}\NormalTok{ mean\_cl\_boot, }\AttributeTok{geom =} \StringTok{"errorbar"}\NormalTok{, }\AttributeTok{width =} \FloatTok{0.2}\NormalTok{, }\AttributeTok{alpha =} \FloatTok{0.5}\NormalTok{, , }\AttributeTok{position=}\FunctionTok{position\_dodge}\NormalTok{(}\AttributeTok{width=}\FloatTok{0.5}\NormalTok{)) }\SpecialCharTok{+}  
  \FunctionTok{stat\_summary}\NormalTok{(}\AttributeTok{fun =}\NormalTok{ mean, }\AttributeTok{geom =} \StringTok{"point"}\NormalTok{, }\AttributeTok{size =} \DecValTok{3}\NormalTok{, }\AttributeTok{position=}\FunctionTok{position\_dodge}\NormalTok{(}\AttributeTok{width=}\FloatTok{0.5}\NormalTok{)) }\SpecialCharTok{+}  
  \FunctionTok{stat\_summary}\NormalTok{(}\AttributeTok{fun =}\NormalTok{ mean, }\AttributeTok{geom =} \StringTok{"line"}\NormalTok{, }\AttributeTok{position=}\FunctionTok{position\_dodge}\NormalTok{(}\AttributeTok{width=}\FloatTok{0.5}\NormalTok{)) }\SpecialCharTok{+}  
  \FunctionTok{scale\_fill\_manual}\NormalTok{(}\AttributeTok{values =}\NormalTok{ groupcolors) }\SpecialCharTok{+} 
  \FunctionTok{scale\_color\_manual}\NormalTok{(}\AttributeTok{values =}\NormalTok{ groupcolors)}\SpecialCharTok{+} 
  \FunctionTok{scale\_y\_continuous}\NormalTok{(}\AttributeTok{labels =}\NormalTok{ scales}\SpecialCharTok{::}\FunctionTok{percent\_format}\NormalTok{(}\AttributeTok{accuracy =} \DecValTok{1}\NormalTok{)) }\SpecialCharTok{+} 
  \FunctionTok{labs}\NormalTok{(}
    \AttributeTok{x =} \StringTok{"Trial"}\NormalTok{,}
    \AttributeTok{y =} \StringTok{"Proportion of repeat choices (\%)"}\NormalTok{,}
    \AttributeTok{title =} \StringTok{"Exploration and exploitation over time"}
\NormalTok{  ) }\SpecialCharTok{+}
  \FunctionTok{theme\_classic}\NormalTok{() }\SpecialCharTok{+}
  \FunctionTok{theme}\NormalTok{(}\AttributeTok{strip.background =} \FunctionTok{element\_blank}\NormalTok{(),  }
        \AttributeTok{strip.text =} \FunctionTok{element\_text}\NormalTok{(}\AttributeTok{color =} \StringTok{"black"}\NormalTok{, }\AttributeTok{size=}\DecValTok{12}\NormalTok{),}
        \AttributeTok{legend.position =} \StringTok{"inside"}\NormalTok{, }
        \AttributeTok{legend.position.inside =} \FunctionTok{c}\NormalTok{(}\FloatTok{0.15}\NormalTok{, }\DecValTok{1}\NormalTok{),   }\CommentTok{\# Use legend.position.inside}
        \AttributeTok{legend.justification =} \FunctionTok{c}\NormalTok{(}\DecValTok{1}\NormalTok{, }\DecValTok{1}\NormalTok{),}
        \AttributeTok{legend.title =} \FunctionTok{element\_blank}\NormalTok{()}
\NormalTok{  )  }
\end{Highlighting}
\end{Shaded}

\begin{center}
\pandocbounded{\includegraphics[keepaspectratio]{gridsearch_parkinson_behavioral_analyses_files/figure-pdf/unnamed-chunk-13-1.pdf}}
\end{center}

\begin{Shaded}
\begin{Highlighting}[]
\FunctionTok{ggsave}\NormalTok{(}\StringTok{"plots/explore{-}exploit\_time.png"}\NormalTok{, }\AttributeTok{dpi=}\DecValTok{300}\NormalTok{, }\AttributeTok{width =} \DecValTok{6}\NormalTok{, }\AttributeTok{height =}\DecValTok{4}\NormalTok{)}
\end{Highlighting}
\end{Shaded}

\section{Spatial trajectories}\label{spatial-trajectories}

We next consider participant's spatial search trajectories (distance
among consecutive clicks). Distance is measured as Manhattan distance
between consecutive clicks, such that repeat clicks have distance 0,
clicking directly neighbouring tiles has distance 1, and clicks further
away have distances \textgreater1.

The most frequent choice was to select a neighboring tile (distance =
1), reflecting a local search approach (Wu et al., 2018). On average,
PNP patients had the shortest distances, indicating more local searches
and repeated clicks. PD+ patients had greater distances than PNP but
shorter than PD- patients, who showed the highest distances. The
distribution of distances shows that this is primarily due to the few
repeat choices (distance = 0) they made, i.e.~very limited exploitation
behavior.

\begin{center}
\pandocbounded{\includegraphics[keepaspectratio]{gridsearch_parkinson_behavioral_analyses_files/figure-pdf/unnamed-chunk-14-1.pdf}}
\end{center}

PNP patients had lower search distances than the PD+ group:
\(t(40)=-2.2\), \(p=.035\), \(d=0.7\), \(BF=1.9\) and lower distances
than the PD- group, \(t(40)=-2.2\), \(p=.035\), \(d=0.7\), \(BF=1.9\)
There was no difference between Parkinson patient with (PD+) and without
(PD-) medication, \(t(40)=-0.8\), \(p=.454\), \(d=0.2\), \(BF=.38\)

\subsection{Types of choices}\label{types-of-choices}

We can also categorize each consecutive click as ``repeat'' (clicking
the same tile as in the previous round), ``near'' (clicking a directly
neighboring tile, i.e.~distance=1), or ``far'' (clicking a tile with
distance \textgreater{} 1). We first computed for each participant the
proportion of type of choices across all 8 rounds x 25 clicks = 200
search decisions and then plot the mean proportion for each group.

The analyses reveal distinct search patterns across patient groups. PNP
participants had the highest proportion of repeat (exploit) decisions,
followed by the PD+ group. The proportion of repeat decisions in the PD-
group was minimal. These behaviors help explain the differences in
learning curves, where PNP patients showed the most significant
improvement, followed by PD+ patients. In contrast, PD- patients
exhibited no improvement across trials, due to their lower tendency to
exploit high-reward options.

\begin{center}
\pandocbounded{\includegraphics[keepaspectratio]{gridsearch_parkinson_behavioral_analyses_files/figure-pdf/unnamed-chunk-16-1.pdf}}
\end{center}

An analysis of consecutive choice types over time reveals clear
differences in search behavior between the groups. Both PNP and PD+
patients adapt their strategies as the round progresses by decreasing
the number of local (distance = 1) and far (distance \textgreater{} 1)
choices, while increasing the number of exploit decisions, indicating a
shift from exploration to exploitation. Notably, the data indicate a
faster shift to exploitation for PNP patients compared to PD+ patients,
with an earlier and stronger preference for re-selecting known
high-reward options. In contrast, PD- patients show limited adaptation,
with the proportions of each decision type remaining relatively stable
throughout the round, aside from a slight increase in exploit decisions.

\begin{Shaded}
\begin{Highlighting}[]
\NormalTok{df\_types\_choices\_trial\_subject }\OtherTok{\textless{}{-}}\NormalTok{ dat }\SpecialCharTok{\%\textgreater{}\%}
    \FunctionTok{filter}\NormalTok{(round }\SpecialCharTok{\%in\%} \DecValTok{2}\SpecialCharTok{:}\DecValTok{9} \SpecialCharTok{\&}\NormalTok{ trial }\SpecialCharTok{\textgreater{}} \DecValTok{0}\NormalTok{) }\SpecialCharTok{\%\textgreater{}\%}
    \FunctionTok{group\_by}\NormalTok{(id, group, trial, type\_choice) }\SpecialCharTok{\%\textgreater{}\%}
    \FunctionTok{summarise}\NormalTok{(}\AttributeTok{n =} \FunctionTok{n}\NormalTok{()) }\SpecialCharTok{\%\textgreater{}\%}
    \FunctionTok{complete}\NormalTok{(type\_choice, }\AttributeTok{fill =} \FunctionTok{list}\NormalTok{(}\AttributeTok{n =} \DecValTok{0}\NormalTok{)) }\SpecialCharTok{\%\textgreater{}\%}  \CommentTok{\# }
    \FunctionTok{group\_by}\NormalTok{(id, group, trial) }\SpecialCharTok{\%\textgreater{}\%}
    \FunctionTok{mutate}\NormalTok{(}\AttributeTok{prop =}\NormalTok{ n }\SpecialCharTok{/} \FunctionTok{sum}\NormalTok{(n)) }\SpecialCharTok{\%\textgreater{}\%}  \CommentTok{\# Calculate proportion for each type\_choice}
    \FunctionTok{ungroup}\NormalTok{()}


\FunctionTok{ggplot}\NormalTok{(df\_types\_choices\_trial\_subject, }\FunctionTok{aes}\NormalTok{(}\AttributeTok{x =}\NormalTok{ trial, }\AttributeTok{y =}\NormalTok{ prop, }\AttributeTok{color =}\NormalTok{ type\_choice, }\AttributeTok{group =}\NormalTok{ type\_choice)) }\SpecialCharTok{+}
  \FunctionTok{facet\_wrap}\NormalTok{(}\SpecialCharTok{\textasciitilde{}}\NormalTok{group) }\SpecialCharTok{+}  
  \FunctionTok{stat\_summary}\NormalTok{(}\AttributeTok{fun.data =}\NormalTok{ mean\_cl\_boot, }\AttributeTok{geom =} \StringTok{"errorbar"}\NormalTok{, }\AttributeTok{width =} \FloatTok{0.2}\NormalTok{, }\AttributeTok{alpha =} \FloatTok{0.5}\NormalTok{, , }\AttributeTok{position=}\FunctionTok{position\_dodge}\NormalTok{(}\AttributeTok{width=}\FloatTok{0.5}\NormalTok{)) }\SpecialCharTok{+}  
  \FunctionTok{stat\_summary}\NormalTok{(}\AttributeTok{fun =}\NormalTok{ mean, }\AttributeTok{geom =} \StringTok{"point"}\NormalTok{, }\AttributeTok{size =} \DecValTok{2}\NormalTok{, }\AttributeTok{position=}\FunctionTok{position\_dodge}\NormalTok{(}\AttributeTok{width=}\FloatTok{0.5}\NormalTok{)) }\SpecialCharTok{+}  
  \FunctionTok{stat\_summary}\NormalTok{(}\AttributeTok{fun =}\NormalTok{ mean, }\AttributeTok{geom =} \StringTok{"line"}\NormalTok{, }\AttributeTok{position=}\FunctionTok{position\_dodge}\NormalTok{(}\AttributeTok{width=}\FloatTok{0.5}\NormalTok{)) }\SpecialCharTok{+}  
  \FunctionTok{scale\_fill\_manual}\NormalTok{(}\AttributeTok{values =}\NormalTok{ groupcolors) }\SpecialCharTok{+} 
  \FunctionTok{scale\_color\_manual}\NormalTok{(}\AttributeTok{values =}\NormalTok{ groupcolors)}\SpecialCharTok{+} 
  \FunctionTok{labs}\NormalTok{(}
    \AttributeTok{x =} \StringTok{"Trial"}\NormalTok{,}
    \AttributeTok{y =} \StringTok{"Mean (±95\% CI)"}\NormalTok{,}
    \AttributeTok{title =} \StringTok{"Types of choices over time"}\NormalTok{,}
    \AttributeTok{color =} \StringTok{"Type choice"}
\NormalTok{  ) }\SpecialCharTok{+}
  \FunctionTok{theme\_classic}\NormalTok{() }\SpecialCharTok{+}
  \FunctionTok{theme}\NormalTok{(}\AttributeTok{strip.background =} \FunctionTok{element\_blank}\NormalTok{(),  }
        \AttributeTok{strip.text =} \FunctionTok{element\_text}\NormalTok{(}\AttributeTok{color =} \StringTok{"black"}\NormalTok{, }\AttributeTok{size=}\DecValTok{12}\NormalTok{),}
        \AttributeTok{legend.position =} \StringTok{"inside"}\NormalTok{, }
        \AttributeTok{legend.position.inside =} \FunctionTok{c}\NormalTok{(}\FloatTok{0.15}\NormalTok{, }\DecValTok{1}\NormalTok{),   }\CommentTok{\# Use legend.position.inside}
        \AttributeTok{legend.justification =} \FunctionTok{c}\NormalTok{(}\DecValTok{1}\NormalTok{, }\DecValTok{1}\NormalTok{),}
        \AttributeTok{legend.title =} \FunctionTok{element\_blank}\NormalTok{(),}
        \AttributeTok{legend.spacing.y =} \FunctionTok{unit}\NormalTok{(}\FloatTok{0.05}\NormalTok{, }\StringTok{\textquotesingle{}cm\textquotesingle{}}\NormalTok{), }
        \AttributeTok{legend.key.height =} \FunctionTok{unit}\NormalTok{(}\FloatTok{0.3}\NormalTok{, }\StringTok{\textquotesingle{}cm\textquotesingle{}}\NormalTok{)  }
\NormalTok{  )    }
\end{Highlighting}
\end{Shaded}

\begin{center}
\pandocbounded{\includegraphics[keepaspectratio]{gridsearch_parkinson_behavioral_analyses_files/figure-pdf/unnamed-chunk-17-1.pdf}}
\end{center}

\begin{Shaded}
\begin{Highlighting}[]
\FunctionTok{ggsave}\NormalTok{(}\StringTok{"plots/types\_choice\_by\_trial.png"}\NormalTok{, }\AttributeTok{dpi=}\DecValTok{300}\NormalTok{, }\AttributeTok{width =} \DecValTok{8}\NormalTok{, }\AttributeTok{height =} \DecValTok{4}\NormalTok{)}
\end{Highlighting}
\end{Shaded}

\subsection{Distance as function of previous
reward}\label{distance-as-function-of-previous-reward}

Finally, we analysed the relation between the value of a reward obtained
at time \(t\) and the search distance on the subsequent trial \(t+1\).
If a large reward was obtained, searchers should search more locally,
while conversely, if a low reward was obtained, searchers should be more
likely to search farther away.

First, we examined the correlations based on the raw data, without
accounting for the hierarchical structure of the dataset. These analyses
revealed a notably stronger correlation for PNP and PD+ patients
compared to PD- patients, suggesting that Parkinson's patients off
medication exhibited less adaptive search behavior than those on
medication and individuals with polyneuropathies

\begin{itemize}
\tightlist
\item
  PNP: \(r=-.46\), \(p<.001\), \(BF>100\)
\item
  PD+: \(r=-.34\), \(p<.001\), \(BF>100\)
\item
  PD-: \(r=-.19\), \(p<.001\), \(BF>100\)
\end{itemize}

\begin{Shaded}
\begin{Highlighting}[]
\CommentTok{\# correlation of previous reward and distance of consecutive choices, by age group and environment}
\CommentTok{\# overall, ignoring within{-}subject factor}
\CommentTok{\# dat \%\textgreater{}\% }
\CommentTok{\#   filter(trial != 0 \& round \%in\% 2:9) \%\textgreater{}\% \# exclude first (randomly revealed) tile and practice round and bonus round}
\CommentTok{\#   group\_by(group) \%\textgreater{}\% }
\CommentTok{\#   summarise(corTestPretty(previous\_reward, distance))}

\CommentTok{\# mean correlation between distance and reward obtained on previous step}
\CommentTok{\# first aggregated within each round and then within each subject}
\CommentTok{\# such that there is one correlation for each subject}

\CommentTok{\# reward\_distance\_cor \textless{}{-} dat \%\textgreater{}\% }
\CommentTok{\#   filter(trial != 0 \& round \%in\% 2:9) \%\textgreater{}\% \# exclude first (randomly revealed) tile and practice round and bonus round}
\CommentTok{\#   group\_by(id, round, group) \%\textgreater{}\% }
\CommentTok{\#   summarise(cor = cor(previous\_reward, distance)) \%\textgreater{}\% }
\CommentTok{\#   mutate(cor = replace\_na(cor, 0)) \%\textgreater{}\%  \# in some rounds subjects clicked the same tile throughout; set cor=0}
\CommentTok{\#   ungroup() \%\textgreater{}\% }
\CommentTok{\#   group\_by(id, group) \%\textgreater{}\% }
\CommentTok{\#   summarise(mean\_cor = mean(cor))}

\CommentTok{\# mean correlation between distance and reward obtained on previous step as function of group}
\CommentTok{\# reward\_distance\_cor \%\textgreater{}\% }
\CommentTok{\#   group\_by(group) \%\textgreater{}\% }
\CommentTok{\#   summarise(n = n(),}
\CommentTok{\#             m\_cor = mean(mean\_cor),}
\CommentTok{\#             SD\_cor = sd(mean\_cor),}
\CommentTok{\#             se\_cor = SD\_cor / sqrt(n),}
\CommentTok{\#             lower\_ci\_cor = m\_cor {-} qt(1 {-} (0.05 / 2), n {-} 1) * se\_cor,}
\CommentTok{\#             upper\_ci\_cor = m\_cor + qt(1 {-} (0.05 / 2), n {-} 1) * se\_cor)}

\CommentTok{\#plot regression lines based on raw data}
\CommentTok{\# ggplot(subset(dat, trial \textgreater{} 0 \& round \%in\% 2:9), aes(x = previous\_reward, y = distance, color = group)) +}
\CommentTok{\#   facet\_wrap(\textasciitilde{}group) +  }
\CommentTok{\#   geom\_jitter(alpha = 0.3, width = 0.1, height = 0.1) +  }
\CommentTok{\#   geom\_smooth(method = "lm", formula = y \textasciitilde{} x, se = TRUE) +  }
\CommentTok{\#   }
\CommentTok{\#   ggtitle("Regression Lines for Distance by Previous Reward and Group") +}
\CommentTok{\#   theme\_minimal() +}
\CommentTok{\#   xlab("Previous Reward") +}
\CommentTok{\#   ylab("Distance")}
\end{Highlighting}
\end{Shaded}

Given the nested structure of the data, we next employed hierarchical
regression analyses to predict search distance based on the reward
obtained in the previous step, group, and their interaction as
population-level (fixed) effects. We accounted for individual
differences by allowing participants to have random intercepts (Note BM:
we can also include random slopes when we have more data points, but
lmer did not converge and brm also need some tweaks).

These analyses show that both the magnitude of reward obtained on the
last step and group influence search distance. Notably, Parkinson
patients off medication (PD-) adapted their search behavior less in
response to reward magnitude, while patients on medication (PD+) exhibit
more adaptation, but still less than the PNP group.

\begin{Shaded}
\begin{Highlighting}[]
\CommentTok{\# for now, random intercepts only, Random intercept + random slope not stable}
\CommentTok{\# lmer\_distance\_reward \textless{}{-} lmer(distance \textasciitilde{} previous\_reward * group + (previous\_reward + group | id), }
\CommentTok{\# data = subset(dat, trial \textgreater{} 0 \& round \%in\% 2:9))}
\CommentTok{\# fit model}
\NormalTok{lmer\_distance\_reward }\OtherTok{\textless{}{-}} \FunctionTok{lmer}\NormalTok{(distance }\SpecialCharTok{\textasciitilde{}}\NormalTok{ previous\_reward }\SpecialCharTok{*}\NormalTok{ group }\SpecialCharTok{+}\NormalTok{ (}\DecValTok{1} \SpecialCharTok{|}\NormalTok{ id), }
                             \AttributeTok{data =} \FunctionTok{subset}\NormalTok{(dat, trial }\SpecialCharTok{\textgreater{}} \DecValTok{0} \SpecialCharTok{\&}\NormalTok{ round }\SpecialCharTok{\%in\%} \DecValTok{2}\SpecialCharTok{:}\DecValTok{9}\NormalTok{))}

\CommentTok{\#summary(lmer\_distance\_reward)}
\CommentTok{\#emmeans(lmer\_distance\_reward, pairwise \textasciitilde{} previous\_reward | group, pbkrtest.limit = 15000)}

\FunctionTok{plot\_model}\NormalTok{(lmer\_distance\_reward, }\AttributeTok{type =} \StringTok{"pred"}\NormalTok{, }\AttributeTok{terms =} \FunctionTok{c}\NormalTok{(}\StringTok{"previous\_reward"}\NormalTok{, }\StringTok{"group"}\NormalTok{)) }\SpecialCharTok{+}
  \FunctionTok{stat\_summary}\NormalTok{(dat, }\AttributeTok{mapping=}\FunctionTok{aes}\NormalTok{(}\AttributeTok{x=}\NormalTok{previous\_reward, }\AttributeTok{y=}\NormalTok{distance, }\AttributeTok{color=}\NormalTok{group, }\AttributeTok{fill=}\NormalTok{group,}\AttributeTok{shape =}\NormalTok{ group), }\AttributeTok{fun=}\NormalTok{mean, }\AttributeTok{geom=}\StringTok{\textquotesingle{}point\textquotesingle{}}\NormalTok{, }\AttributeTok{alpha=}\FloatTok{0.7}\NormalTok{, }\AttributeTok{size=}\FloatTok{0.5}\NormalTok{, }\AttributeTok{na.rm =} \ConstantTok{TRUE}\NormalTok{)}\SpecialCharTok{+}
  \FunctionTok{scale\_x\_continuous}\NormalTok{(}\StringTok{\textquotesingle{}Previous Reward\textquotesingle{}}\NormalTok{, }\AttributeTok{breaks =}\NormalTok{ (}\FunctionTok{c}\NormalTok{(}\DecValTok{0}\NormalTok{,}\DecValTok{10}\NormalTok{,}\DecValTok{20}\NormalTok{,}\DecValTok{30}\NormalTok{,}\DecValTok{40}\NormalTok{,}\DecValTok{50}\NormalTok{))) }\SpecialCharTok{+}
  \FunctionTok{ylab}\NormalTok{(}\StringTok{\textquotesingle{}Distance to Next Option\textquotesingle{}}\NormalTok{)}\SpecialCharTok{+}
  \FunctionTok{scale\_fill\_manual}\NormalTok{(}\AttributeTok{values=}\NormalTok{groupcolors) }\SpecialCharTok{+}
  \FunctionTok{scale\_color\_manual}\NormalTok{(}\AttributeTok{values=}\NormalTok{groupcolors) }\SpecialCharTok{+}
  \FunctionTok{ggtitle}\NormalTok{(}\StringTok{\textquotesingle{}Search Distance \textasciitilde{} Previous Reward (lmer)\textquotesingle{}}\NormalTok{) }\SpecialCharTok{+}
  \FunctionTok{theme\_classic}\NormalTok{() }\SpecialCharTok{+}
  \FunctionTok{theme}\NormalTok{(}\AttributeTok{legend.position =} \StringTok{"inside"}\NormalTok{, }
        \AttributeTok{legend.position.inside =} \FunctionTok{c}\NormalTok{(}\FloatTok{0.85}\NormalTok{, }\FloatTok{0.9}\NormalTok{),   }\CommentTok{\# Use legend.position.inside}
        \AttributeTok{legend.justification =} \FunctionTok{c}\NormalTok{(}\DecValTok{1}\NormalTok{, }\DecValTok{1}\NormalTok{),}
        \AttributeTok{legend.title =} \FunctionTok{element\_blank}\NormalTok{(),}
        \AttributeTok{legend.box.background =}  \FunctionTok{element\_blank}\NormalTok{())     }
\end{Highlighting}
\end{Shaded}

\begin{center}
\pandocbounded{\includegraphics[keepaspectratio]{gridsearch_parkinson_behavioral_analyses_files/figure-pdf/unnamed-chunk-19-1.pdf}}
\end{center}

\begin{Shaded}
\begin{Highlighting}[]
\FunctionTok{ggsave}\NormalTok{(}\StringTok{"plots/regression\_distance\_reward\_lmer.png"}\NormalTok{, }\AttributeTok{dpi=}\DecValTok{300}\NormalTok{, }\AttributeTok{height=}\DecValTok{3}\NormalTok{, }\AttributeTok{width=}\DecValTok{4}\NormalTok{)}
\end{Highlighting}
\end{Shaded}

Results of a Bayesian hierarchical regression with random intercepts and
random slopes for subjects yield comparable findings, again indicating
distinct search patterns and responses to reward magnitude across
groups, with the PD- group exhibiting less adaptive behavior.

\begin{Shaded}
\begin{Highlighting}[]
\CommentTok{\# Bayesian regression analysis}
\CommentTok{\# run\_model() is a wrapper for brm models such that it saves the full model the first time it is run, otherwise it loads it from disk from directory \textasciigrave{}\textasciitilde{}brm\textasciigrave{}}
\CommentTok{\# Fixed effects: previous\_reward and group.}
\CommentTok{\# Random effects: random slopes and a random intercept for both previous\_reward and group by id, i.e., the effect of previous\_reward and group can vary across individuals (id).}

\CommentTok{\# random intercept and random slope}
\CommentTok{\# brm\_distance\_reward \textless{}{-} run\_model(brm(distance \textasciitilde{} previous\_reward * group + (previous\_reward + group | id), }

\CommentTok{\# random intercept                                     }
\NormalTok{brm\_distance\_reward }\OtherTok{\textless{}{-}} \FunctionTok{run\_model}\NormalTok{(}\FunctionTok{brm}\NormalTok{(distance }\SpecialCharTok{\textasciitilde{}}\NormalTok{ previous\_reward }\SpecialCharTok{*}\NormalTok{ group }\SpecialCharTok{+}\NormalTok{ (}\DecValTok{1}\SpecialCharTok{|}\NormalTok{id), }
                                     \AttributeTok{data=}\FunctionTok{subset}\NormalTok{(dat, trial }\SpecialCharTok{\textgreater{}} \DecValTok{0} \SpecialCharTok{\&}\NormalTok{ round }\SpecialCharTok{\%in\%} \DecValTok{2}\SpecialCharTok{:}\DecValTok{9}\NormalTok{ ), }
                                     \AttributeTok{cores=}\DecValTok{4}\NormalTok{,  }\CommentTok{\# running into problems with cores \textgreater{} 1}
                                     \AttributeTok{seed =} \DecValTok{0815}\NormalTok{,}
                                     \AttributeTok{iter =} \DecValTok{5000}\NormalTok{,}
                                     \AttributeTok{warmup=}\DecValTok{1000}\NormalTok{,}
                                     \AttributeTok{control =} \FunctionTok{list}\NormalTok{(}\AttributeTok{adapt\_delta =} \FloatTok{0.99}\NormalTok{, }\AttributeTok{max\_treedepth =} \DecValTok{15}\NormalTok{)),}
                                 \CommentTok{\#prior = prior(normal(0,10), class = "b")), }
                                 \AttributeTok{modelName =} \StringTok{\textquotesingle{}brm\_distance\_reward\textquotesingle{}}\NormalTok{)}
\CommentTok{\#tab\_model(brm\_distance\_reward, bpe="mean", title = "Bayesian regression results: Search distance as function of reward on previous step.") }
\CommentTok{\#bayes\_R2(brm\_distance\_reward) }
\end{Highlighting}
\end{Shaded}

\begin{Shaded}
\begin{Highlighting}[]
\CommentTok{\# Compute marginal effects for distance as a function of previous\_reward across groups}
\CommentTok{\# marginal\_effects\_data \textless{}{-} conditional\_effects(brm\_distance\_reward, effects = "previous\_reward:group") + theme\_classic()}
\CommentTok{\# }
\CommentTok{\# \# Generate conditional effects for previous\_reward by group}
\CommentTok{\# effects \textless{}{-} conditional\_effects(brm\_distance\_reward, effects = "previous\_reward:group")}
\CommentTok{\# }
\CommentTok{\# \# Extract the plot for previous\_reward:group}
\CommentTok{\# plot\_effect \textless{}{-} plot(effects, plot = FALSE)[[1]]}
\CommentTok{\# }
\CommentTok{\# plot\_effect +}
\CommentTok{\#    xlab(\textquotesingle{}Previous Reward\textquotesingle{})+}
\CommentTok{\#   ylab(\textquotesingle{}Distance to Next Option\textquotesingle{})+}
\CommentTok{\#   scale\_color\_viridis(discrete=TRUE, name=\textquotesingle{}Group\textquotesingle{}, direction=1) +}
\CommentTok{\#   scale\_fill\_viridis(discrete=TRUE, name=\textquotesingle{}Group\textquotesingle{}, direction=1) +}
\CommentTok{\#   ggtitle(\textquotesingle{}Search Distance \textasciitilde{} Previous Reward\textquotesingle{}) +}
\CommentTok{\#   theme\_classic() +}
\CommentTok{\#   theme(legend.position = "inside", }
\CommentTok{\#         legend.position.inside = c(0.85, 0.9),   \# Use legend.position.inside}
\CommentTok{\#         legend.justification = c(1, 1),}
\CommentTok{\#         legend.title = element\_blank())           \# Align top{-}right corner of legend}


\CommentTok{\# generate plot manually  predictions (otherwise difficult to plot the mean empirical values per geom\_point)}
\NormalTok{prevReward }\OtherTok{\textless{}{-}}  \FunctionTok{seq}\NormalTok{(}\DecValTok{0}\NormalTok{,}\DecValTok{50}\NormalTok{) }\CommentTok{\#/ 50 \# normalized reward}
\NormalTok{group  }\OtherTok{\textless{}{-}}  \FunctionTok{levels}\NormalTok{(dat}\SpecialCharTok{$}\NormalTok{group)}
\NormalTok{newdat }\OtherTok{\textless{}{-}}  \FunctionTok{expand.grid}\NormalTok{(}\AttributeTok{previous\_reward=}\NormalTok{prevReward, }\AttributeTok{group=}\NormalTok{group)}

\CommentTok{\# predict distance based on previous reward}
\NormalTok{preds }\OtherTok{\textless{}{-}}  \FunctionTok{fitted}\NormalTok{(brm\_distance\_reward, }\AttributeTok{re\_formula=}\ConstantTok{NA}\NormalTok{, }\AttributeTok{newdata=}\NormalTok{newdat, }\AttributeTok{probs=}\FunctionTok{c}\NormalTok{(.}\DecValTok{025}\NormalTok{, .}\DecValTok{975}\NormalTok{))}
\NormalTok{predsDF }\OtherTok{\textless{}{-}}  \FunctionTok{data.frame}\NormalTok{(}\AttributeTok{previous\_reward=}\FunctionTok{rep}\NormalTok{(prevReward, }\DecValTok{3}\NormalTok{),}
                       \AttributeTok{group=}\FunctionTok{rep}\NormalTok{(}\FunctionTok{levels}\NormalTok{(dat}\SpecialCharTok{$}\NormalTok{group), }\AttributeTok{each=}\FunctionTok{length}\NormalTok{(prevReward)),}
                       \AttributeTok{distance=}\NormalTok{preds[,}\DecValTok{1}\NormalTok{],}
                       \AttributeTok{lower=}\NormalTok{preds[,}\DecValTok{3}\NormalTok{],}
                       \AttributeTok{upper=}\NormalTok{preds[,}\DecValTok{4}\NormalTok{])}

\CommentTok{\# average distance}
\NormalTok{grid  }\OtherTok{\textless{}{-}}  \FunctionTok{expand.grid}\NormalTok{(}\AttributeTok{x1=}\DecValTok{0}\SpecialCharTok{:}\DecValTok{7}\NormalTok{, }\AttributeTok{x2=}\DecValTok{0}\SpecialCharTok{:}\DecValTok{7}\NormalTok{, }\AttributeTok{y1=}\DecValTok{0}\SpecialCharTok{:}\DecValTok{7}\NormalTok{, }\AttributeTok{y2=}\DecValTok{0}\SpecialCharTok{:}\DecValTok{7}\NormalTok{)}
\NormalTok{grid}\SpecialCharTok{$}\NormalTok{distance }\OtherTok{\textless{}{-}}  \ConstantTok{NA}

\ControlFlowTok{for}\NormalTok{(i }\ControlFlowTok{in} \DecValTok{1}\SpecialCharTok{:}\FunctionTok{dim}\NormalTok{(grid)[}\DecValTok{1}\NormalTok{])\{}
\NormalTok{  grid}\SpecialCharTok{$}\NormalTok{distance[i] }\OtherTok{\textless{}{-}} \FunctionTok{dist}\NormalTok{(}\FunctionTok{rbind}\NormalTok{(}\FunctionTok{c}\NormalTok{(grid}\SpecialCharTok{$}\NormalTok{x1[i], grid}\SpecialCharTok{$}\NormalTok{x2[i]), }\FunctionTok{c}\NormalTok{(grid}\SpecialCharTok{$}\NormalTok{y1[i], grid}\SpecialCharTok{$}\NormalTok{y2[i])), }\AttributeTok{method =} \StringTok{"manhattan"}\NormalTok{)}
\NormalTok{\}}

\NormalTok{meanDist  }\OtherTok{\textless{}{-}}  \FunctionTok{mean}\NormalTok{(grid}\SpecialCharTok{$}\NormalTok{distance)}

\CommentTok{\# plot predictions}
\FunctionTok{ggplot}\NormalTok{() }\SpecialCharTok{+}
  \FunctionTok{stat\_summary}\NormalTok{(dat, }\AttributeTok{mapping=}\FunctionTok{aes}\NormalTok{(}\AttributeTok{x=}\NormalTok{previous\_reward, }\AttributeTok{y=}\NormalTok{distance, }\AttributeTok{color=}\NormalTok{group, }\AttributeTok{fill=}\NormalTok{group), }\AttributeTok{fun=}\NormalTok{mean, }\AttributeTok{geom=}\StringTok{\textquotesingle{}point\textquotesingle{}}\NormalTok{, }\AttributeTok{alpha=}\FloatTok{0.7}\NormalTok{, }\AttributeTok{size=}\FloatTok{0.5}\NormalTok{, }\AttributeTok{na.rm=}\NormalTok{T)}\SpecialCharTok{+}
  \FunctionTok{geom\_line}\NormalTok{(predsDF, }\AttributeTok{mapping=}\FunctionTok{aes}\NormalTok{(}\AttributeTok{x=}\NormalTok{previous\_reward, }\AttributeTok{y=}\NormalTok{distance, }\AttributeTok{color=}\NormalTok{group), }\AttributeTok{linewidth=}\DecValTok{1}\NormalTok{) }\SpecialCharTok{+}
  \FunctionTok{geom\_ribbon}\NormalTok{(predsDF, }\AttributeTok{mapping=}\FunctionTok{aes}\NormalTok{(}\AttributeTok{x=}\NormalTok{previous\_reward, }\AttributeTok{y=}\NormalTok{distance, }\AttributeTok{ymin=}\NormalTok{lower, }\AttributeTok{ymax=}\NormalTok{upper, }\AttributeTok{fill=}\NormalTok{group), }\AttributeTok{alpha=}\NormalTok{.}\DecValTok{3}\NormalTok{) }\SpecialCharTok{+}
  \CommentTok{\#geom\_hline(yintercept=meanDist, linetype=\textquotesingle{}dashed\textquotesingle{}, color=\textquotesingle{}red\textquotesingle{}) + \# mean distance}
  \CommentTok{\# xlab(\textquotesingle{}Normalized Previous Reward\textquotesingle{})+}
  \FunctionTok{xlab}\NormalTok{(}\StringTok{\textquotesingle{}Previous Reward\textquotesingle{}}\NormalTok{)}\SpecialCharTok{+}
  \FunctionTok{ylab}\NormalTok{(}\StringTok{\textquotesingle{}Distance to Next Option\textquotesingle{}}\NormalTok{)}\SpecialCharTok{+}
  \FunctionTok{scale\_fill\_manual}\NormalTok{(}\AttributeTok{values=}\NormalTok{groupcolors) }\SpecialCharTok{+}
  \FunctionTok{scale\_color\_manual}\NormalTok{(}\AttributeTok{values=}\NormalTok{groupcolors) }\SpecialCharTok{+}
  \FunctionTok{ggtitle}\NormalTok{(}\StringTok{\textquotesingle{}Search Distance \textasciitilde{} Previous Reward (brm)\textquotesingle{}}\NormalTok{) }\SpecialCharTok{+}
  \FunctionTok{theme\_classic}\NormalTok{() }\SpecialCharTok{+}
  \FunctionTok{theme}\NormalTok{(}\AttributeTok{legend.position =} \StringTok{"inside"}\NormalTok{, }
        \AttributeTok{legend.position.inside =} \FunctionTok{c}\NormalTok{(}\FloatTok{0.85}\NormalTok{, }\FloatTok{0.9}\NormalTok{),   }
        \AttributeTok{legend.justification =} \FunctionTok{c}\NormalTok{(}\DecValTok{1}\NormalTok{, }\DecValTok{1}\NormalTok{),}
        \AttributeTok{legend.title =} \FunctionTok{element\_blank}\NormalTok{())          }
\end{Highlighting}
\end{Shaded}

\begin{center}
\pandocbounded{\includegraphics[keepaspectratio]{gridsearch_parkinson_behavioral_analyses_files/figure-pdf/unnamed-chunk-21-1.pdf}}
\end{center}

\begin{Shaded}
\begin{Highlighting}[]
\FunctionTok{ggsave}\NormalTok{(}\StringTok{"plots/regression\_distance\_reward\_brms.png"}\NormalTok{, }\AttributeTok{dpi=}\DecValTok{300}\NormalTok{, }\AttributeTok{height=}\DecValTok{3}\NormalTok{, }\AttributeTok{width=}\DecValTok{4}\NormalTok{)}
\end{Highlighting}
\end{Shaded}

\section{Bonus round judgments}\label{bonus-round-judgments}

In the bonus round, participants made 15 search decisions and then
predicted the rewards for 5 randomly chosen, previously unobserved
tiles. Subsequently, they chose one of the five tiles and continued the
round until the search horizon of 25 clicks was met.

Data frame \emph{dat\_bonus} contains the following variables:

\begin{itemize}
\tightlist
\item
  \emph{id}: participant id
\item
  \emph{bonus\_env\_number}: internal id of the bonus round environment
\item
  \emph{bonus\_environment}: recodes \emph{condition} as \emph{Smooth}
  (high spatial correlation)
\item
  \emph{x} and \emph{y} are the coordinates of the random tiles on the
  grid for whcih participants were asked to provide reward estimates
\item
  \emph{givenValue}: participant reward judgment (scale 0-50)
\item
  \emph{howSecure}: participant confidence for given reward judgment
  (scale 0-10)
\item
  \emph{chosen\_x} and \emph{chosen\_y} are the coordinates of the tile
  chose after making reward and confidence judgments for 5 random tiles
\item
  \emph{true\_z} is the ground truth, i.e.~true expected reward of a
  tile
\item
  \emph{error} is the absolute deviation between participants reward
  estimates (\emph{givenValue}) and ground truth (\emph{true\_z})
\item
  \emph{chosen} is whether the option was chosen or not (participants
  chose one of the five options after estimating their value and
  confidence in their reward prediction)
\end{itemize}

\begin{tcolorbox}[enhanced jigsaw, bottomtitle=1mm, leftrule=.75mm, colback=white, toptitle=1mm, breakable, title=\textcolor{quarto-callout-note-color}{\faInfo}\hspace{0.5em}{Note}, colbacktitle=quarto-callout-note-color!10!white, opacitybacktitle=0.6, coltitle=black, colframe=quarto-callout-note-color-frame, left=2mm, opacityback=0, bottomrule=.15mm, rightrule=.15mm, titlerule=0mm, arc=.35mm, toprule=.15mm]

Charley: is this scaling still correct (taken from YKWG code)?
\texttt{bonus\_environment\$z\ \textless{}-\ bonus\_environment\$z\ *\ scale\_factor\ +\ 5}

\end{tcolorbox}

\begin{tabular}{r|r|l|r|r|r|r|r|r|r|l|r|l}
\hline
id & bonus\_env\_number & bonus\_environment & x & y & givenValue & howSecure & chosen\_x & chosen\_y & true\_z & chosen & error & group\\
\hline
111 & 38 & Smooth & 5 & 6 & 20 & 5 & 7 & 3 & 24.98 & not chosen & 4.98 & PNP\\
\hline
111 & 38 & Smooth & 2 & 7 & 26 & 4 & 7 & 3 & 6.51 & not chosen & 19.49 & PNP\\
\hline
111 & 38 & Smooth & 7 & 3 & 16 & 5 & 7 & 3 & 38.06 & chosen & 22.06 & PNP\\
\hline
111 & 38 & Smooth & 0 & 7 & 28 & 3 & 7 & 3 & 6.55 & not chosen & 21.45 & PNP\\
\hline
111 & 38 & Smooth & 7 & 6 & 30 & 5 & 7 & 3 & 41.59 & not chosen & 11.59 & PNP\\
\hline
115 & 39 & Smooth & 0 & 0 & 19 & 4 & 3 & 1 & 38.01 & not chosen & 19.01 & PNP\\
\hline
\end{tabular}

\subsection{Prediction error}\label{prediction-error}

The plot shows the mean absolute error between participants' estimates
and the true underlying expected reward, for each age group and
environment.

\begin{center}
\pandocbounded{\includegraphics[keepaspectratio]{gridsearch_parkinson_behavioral_analyses_files/figure-pdf/unnamed-chunk-23-1.pdf}}
\end{center}

Compared to a random baseline, all groups performed better than chance
level:

\begin{itemize}
\tightlist
\item
  PNP: \(t(19)=-2.2\), \(p=.040\), \(d=0.5\), \(BF=1.7\)
\item
  PD+: \(t(21)=-7.5\), \(p<.001\), \(d=1.6\), \(BF>100\)
\item
  PD-: \(t(19)=-4.7\), \(p<.001\), \(d=1.0\), \(BF>100\)
\end{itemize}

There was no difference between Parkinson patients on and off
medication. PNP participants has had higher prediction error than
Parkinson patients on medication (PD+), but were not worse than
Parkinson participants off medication: (PD-).

\subsection{Prediction error and
confidence}\label{prediction-error-and-confidence}

\begin{Shaded}
\begin{Highlighting}[]
\CommentTok{\# Across all judgments and participants, there was no systematic relation between confidence and prediction error:}
\CommentTok{\# corTestPretty(dat\_bonus$error, dat\_bonus$howSecure, method = "kendall") }
\CommentTok{\# cor.test(dat\_bonus$error, dat\_bonus$howSecure, method = "kendall") }
\CommentTok{\# correlationBF(dat\_bonus$error, dat\_bonus$howSecure, method = "kendall") }
\end{Highlighting}
\end{Shaded}

A Bayesian regression with with prediction error as dependent variable,
and confidence and group and their interaction as population-level
(``fixed'') effects, and a random intercept for participants showed that
for PNP patients confidence and predictione rror were negatively
correlated (i.e., lower confidence was associated with a higher error),
whereas for the two Parkinson groups there was no relation.

\begin{Shaded}
\begin{Highlighting}[]
\NormalTok{brm\_bonus\_confidence\_error\_by\_group }\OtherTok{\textless{}{-}} \FunctionTok{run\_model}\NormalTok{(}\FunctionTok{brm}\NormalTok{(error }\SpecialCharTok{\textasciitilde{}}\NormalTok{ howSecure }\SpecialCharTok{*}\NormalTok{ group }\SpecialCharTok{+}\NormalTok{ (}\DecValTok{1}\SpecialCharTok{|}\NormalTok{id), }
                                                     \AttributeTok{data=}\NormalTok{dat\_bonus, }
                                                     \AttributeTok{cores=}\DecValTok{4}\NormalTok{,  }
                                                     \AttributeTok{control =} \FunctionTok{list}\NormalTok{(}\AttributeTok{adapt\_delta =} \FloatTok{0.99}\NormalTok{),}
                                                     \AttributeTok{seed =} \DecValTok{0815}\NormalTok{), }
                                                 \AttributeTok{modelName =} \StringTok{\textquotesingle{}brm\_bonus\_confidence\_error\_by\_group\textquotesingle{}}\NormalTok{)}
\CommentTok{\#tab\_model(brm\_bonus\_confidence\_error\_by\_group, bpe = "mean", title = "Bayesian regression results: Prediction error and confidence") }
\CommentTok{\#bayes\_R2(brm\_bonus\_confidence\_error\_by\_group)}
\end{Highlighting}
\end{Shaded}

\begin{center}
\pandocbounded{\includegraphics[keepaspectratio]{gridsearch_parkinson_behavioral_analyses_files/figure-pdf/unnamed-chunk-28-1.pdf}}
\end{center}

\subsection{Analysis of selected
tiles}\label{analysis-of-selected-tiles}

To analyze selected and not-selected options, we first averaged the
predicted reward and confidence of the not-chosen tiles within subjects,
and then compared chosen and not chosen options. Selected tiles tended
to have higher predicted rewards. Participants were not more confident
in selected options, and selected tiles did not have a higher true
reward than not selected tiles.

\begin{Shaded}
\begin{Highlighting}[]
\CommentTok{\# average not{-}chosen tiles within subjects first}
\NormalTok{df\_chosen\_overall }\OtherTok{\textless{}{-}}\NormalTok{ dat\_bonus }\SpecialCharTok{\%\textgreater{}\%} 
  \FunctionTok{group\_by}\NormalTok{(id, chosen) }\SpecialCharTok{\%\textgreater{}\%} 
  \FunctionTok{summarise}\NormalTok{(}\AttributeTok{m\_givenValue =} \FunctionTok{mean}\NormalTok{(givenValue),}
            \AttributeTok{m\_howSecure =} \FunctionTok{mean}\NormalTok{(howSecure),}
            \AttributeTok{m\_true\_z =} \FunctionTok{mean}\NormalTok{(true\_z))}

\CommentTok{\# df\_chosen\_overall \%\textgreater{}\% }
\CommentTok{\#   group\_by(chosen) \%\textgreater{}\% }
\CommentTok{\#   summarise(predicted\_reward = mean(m\_givenValue),}
\CommentTok{\#             confidence = mean(m\_howSecure),}
\CommentTok{\#             true\_reward = mean(m\_true\_z)) \%\textgreater{}\% }
\CommentTok{\#   kable(format = "html", escape = F, digits = 2) \%\textgreater{}\%}
\CommentTok{\#   kable\_styling("striped", full\_width = F)}

\NormalTok{df\_chosen\_group }\OtherTok{\textless{}{-}}\NormalTok{ dat\_bonus }\SpecialCharTok{\%\textgreater{}\%} 
  \FunctionTok{group\_by}\NormalTok{(id, group, chosen) }\SpecialCharTok{\%\textgreater{}\%} 
  \FunctionTok{summarise}\NormalTok{(}\AttributeTok{m\_givenValue =} \FunctionTok{mean}\NormalTok{(givenValue),}
            \AttributeTok{m\_howSecure =} \FunctionTok{mean}\NormalTok{(howSecure),}
            \AttributeTok{m\_true\_z =} \FunctionTok{mean}\NormalTok{(true\_z))}

\CommentTok{\# df\_chosen\_group \%\textgreater{}\% }
\CommentTok{\#   group\_by(group,chosen) \%\textgreater{}\% }
\CommentTok{\#   summarise(predicted\_reward = mean(m\_givenValue),}
\CommentTok{\#             confidence = mean(m\_howSecure),}
\CommentTok{\#             true\_reward = mean(m\_true\_z)) \%\textgreater{}\% }
\CommentTok{\#   kable(format = "html", escape = F, digits = 2) \%\textgreater{}\%}
\CommentTok{\#   kable\_styling("striped", full\_width = F)}

\CommentTok{\# chosen vs not chosen: predicted reward}

\FunctionTok{ggboxplot}\NormalTok{(df\_chosen\_group, }\AttributeTok{x =} \StringTok{"chosen"}\NormalTok{, }\AttributeTok{y =} \StringTok{"m\_givenValue"}\NormalTok{,}
          \AttributeTok{color =} \StringTok{"group"}\NormalTok{, }\AttributeTok{palette =}\NormalTok{groupcolors, }\AttributeTok{fill =} \StringTok{"group"}\NormalTok{, }\AttributeTok{alpha =} \FloatTok{0.2}\NormalTok{,}
          \AttributeTok{add =} \StringTok{"jitter"}\NormalTok{, }\AttributeTok{shape =} \StringTok{"group"}\NormalTok{, }\AttributeTok{title =} \StringTok{"Predicted reward of chosen vs. not chosen options"}\NormalTok{,}
          \AttributeTok{facet.by =} \StringTok{"group"}\NormalTok{) }\SpecialCharTok{+}
  \FunctionTok{ylab}\NormalTok{(}\StringTok{"Predicted reward"}\NormalTok{) }\SpecialCharTok{+}
  \FunctionTok{xlab}\NormalTok{(}\StringTok{""}\NormalTok{) }\SpecialCharTok{+}
  \FunctionTok{stat\_compare\_means}\NormalTok{(}\AttributeTok{comparisons =} \FunctionTok{list}\NormalTok{( }\FunctionTok{c}\NormalTok{(}\StringTok{"chosen"}\NormalTok{, }\StringTok{"not chosen"}\NormalTok{) ), }\AttributeTok{paired =}\NormalTok{ T, }\AttributeTok{method =} \StringTok{"t.test"}\NormalTok{, }\AttributeTok{label =} \StringTok{"p.format"}\NormalTok{) }\SpecialCharTok{+}
  \FunctionTok{stat\_summary}\NormalTok{(}\AttributeTok{fun =}\NormalTok{ mean, }\AttributeTok{geom=}\StringTok{"point"}\NormalTok{, }\AttributeTok{shape =} \DecValTok{23}\NormalTok{, }\AttributeTok{fill =} \StringTok{"white"}\NormalTok{, }\AttributeTok{size=}\DecValTok{3}\NormalTok{) }\SpecialCharTok{+}
  \FunctionTok{theme\_classic}\NormalTok{() }\SpecialCharTok{+}
  \FunctionTok{theme}\NormalTok{(}\AttributeTok{strip.background =} \FunctionTok{element\_blank}\NormalTok{(),  }
        \AttributeTok{strip.text =} \FunctionTok{element\_text}\NormalTok{(}\AttributeTok{color =} \StringTok{"black"}\NormalTok{, }\AttributeTok{size=}\DecValTok{12}\NormalTok{),}
        \AttributeTok{legend.title =} \FunctionTok{element\_blank}\NormalTok{()}
\NormalTok{  )}
\end{Highlighting}
\end{Shaded}

\begin{center}
\pandocbounded{\includegraphics[keepaspectratio]{gridsearch_parkinson_behavioral_analyses_files/figure-pdf/unnamed-chunk-29-1.pdf}}
\end{center}

\begin{Shaded}
\begin{Highlighting}[]
\FunctionTok{ggsave}\NormalTok{(}\StringTok{"plots/bonusround\_chosen\_not\_chosen\_options\_predicted\_reward.png"}\NormalTok{, }\AttributeTok{dpi=}\DecValTok{300}\NormalTok{, }\AttributeTok{width=}\DecValTok{7}\NormalTok{, }\AttributeTok{height =} \DecValTok{5}\NormalTok{)}

\CommentTok{\# chosen vs not chosen: true reward}

\FunctionTok{ggboxplot}\NormalTok{(df\_chosen\_group, }\AttributeTok{x =} \StringTok{"chosen"}\NormalTok{, }\AttributeTok{y =} \StringTok{"m\_true\_z"}\NormalTok{,}
          \AttributeTok{color =} \StringTok{"group"}\NormalTok{, }\AttributeTok{palette =}\NormalTok{groupcolors, }\AttributeTok{fill =} \StringTok{"group"}\NormalTok{, }\AttributeTok{alpha =} \FloatTok{0.2}\NormalTok{,}
          \AttributeTok{add =} \StringTok{"jitter"}\NormalTok{, }\AttributeTok{shape =} \StringTok{"group"}\NormalTok{, }\AttributeTok{title =} \StringTok{"True reward of chosen vs. not chosen options"}\NormalTok{,}
          \AttributeTok{facet.by =} \StringTok{"group"}\NormalTok{) }\SpecialCharTok{+}
  \FunctionTok{ylab}\NormalTok{(}\StringTok{"True reward"}\NormalTok{) }\SpecialCharTok{+}
  \FunctionTok{xlab}\NormalTok{(}\StringTok{""}\NormalTok{) }\SpecialCharTok{+}
  \FunctionTok{stat\_compare\_means}\NormalTok{(}\AttributeTok{comparisons =} \FunctionTok{list}\NormalTok{( }\FunctionTok{c}\NormalTok{(}\StringTok{"chosen"}\NormalTok{, }\StringTok{"not chosen"}\NormalTok{) ), }\AttributeTok{paired =}\NormalTok{ T, }\AttributeTok{method =} \StringTok{"t.test"}\NormalTok{, }\AttributeTok{label =} \StringTok{"p.format"}\NormalTok{) }\SpecialCharTok{+}
  \FunctionTok{stat\_summary}\NormalTok{(}\AttributeTok{fun =}\NormalTok{ mean, }\AttributeTok{geom=}\StringTok{"point"}\NormalTok{, }\AttributeTok{shape =} \DecValTok{23}\NormalTok{, }\AttributeTok{fill =} \StringTok{"white"}\NormalTok{, }\AttributeTok{size=}\DecValTok{3}\NormalTok{) }\SpecialCharTok{+}
  \FunctionTok{theme\_classic}\NormalTok{() }\SpecialCharTok{+}
  \FunctionTok{theme}\NormalTok{(}\AttributeTok{strip.background =} \FunctionTok{element\_blank}\NormalTok{(),  }
        \AttributeTok{strip.text =} \FunctionTok{element\_text}\NormalTok{(}\AttributeTok{color =} \StringTok{"black"}\NormalTok{, }\AttributeTok{size=}\DecValTok{12}\NormalTok{),}
        \AttributeTok{legend.title =} \FunctionTok{element\_blank}\NormalTok{()}
\NormalTok{  )}
\end{Highlighting}
\end{Shaded}

\begin{center}
\pandocbounded{\includegraphics[keepaspectratio]{gridsearch_parkinson_behavioral_analyses_files/figure-pdf/unnamed-chunk-29-2.pdf}}
\end{center}

\begin{Shaded}
\begin{Highlighting}[]
\FunctionTok{ggsave}\NormalTok{(}\StringTok{"plots/bonusround\_chosen\_not\_chosen\_options\_predicted\_reward.png"}\NormalTok{, }\AttributeTok{dpi=}\DecValTok{300}\NormalTok{, }\AttributeTok{width=}\DecValTok{7}\NormalTok{, }\AttributeTok{height =} \DecValTok{5}\NormalTok{)}

\CommentTok{\# chosen vs not chosen: confidence}


\FunctionTok{ggboxplot}\NormalTok{(df\_chosen\_group, }\AttributeTok{x =} \StringTok{"chosen"}\NormalTok{, }\AttributeTok{y =} \StringTok{"m\_howSecure"}\NormalTok{,}
          \AttributeTok{color =} \StringTok{"group"}\NormalTok{, }\AttributeTok{palette =}\NormalTok{groupcolors, }\AttributeTok{fill =} \StringTok{"group"}\NormalTok{, }\AttributeTok{alpha =} \FloatTok{0.2}\NormalTok{,}
          \AttributeTok{add =} \StringTok{"jitter"}\NormalTok{, }\AttributeTok{shape =} \StringTok{"group"}\NormalTok{, }\AttributeTok{title =} \StringTok{"Confidence of chosen vs. not chosen options"}\NormalTok{,}
          \AttributeTok{facet.by =} \StringTok{"group"}\NormalTok{) }\SpecialCharTok{+}
  \FunctionTok{ylab}\NormalTok{(}\StringTok{"Confidence in reward prediction"}\NormalTok{) }\SpecialCharTok{+}
  \FunctionTok{xlab}\NormalTok{(}\StringTok{""}\NormalTok{) }\SpecialCharTok{+}
  \FunctionTok{stat\_compare\_means}\NormalTok{(}\AttributeTok{comparisons =}  \FunctionTok{list}\NormalTok{( }\FunctionTok{c}\NormalTok{(}\StringTok{"chosen"}\NormalTok{, }\StringTok{"not chosen"}\NormalTok{) ), }\AttributeTok{paired =}\NormalTok{ T, }\AttributeTok{method =} \StringTok{"t.test"}\NormalTok{, }\AttributeTok{label =} \StringTok{"p.format"}\NormalTok{) }\SpecialCharTok{+}
  \FunctionTok{stat\_summary}\NormalTok{(}\AttributeTok{fun =}\NormalTok{ mean, }\AttributeTok{geom=}\StringTok{"point"}\NormalTok{, }\AttributeTok{shape =} \DecValTok{23}\NormalTok{, }\AttributeTok{fill =} \StringTok{"white"}\NormalTok{, }\AttributeTok{size=}\DecValTok{3}\NormalTok{) }\SpecialCharTok{+}
  \FunctionTok{theme\_classic}\NormalTok{() }\SpecialCharTok{+}
  \FunctionTok{theme}\NormalTok{(}\AttributeTok{strip.background =} \FunctionTok{element\_blank}\NormalTok{(),  }
        \AttributeTok{strip.text =} \FunctionTok{element\_text}\NormalTok{(}\AttributeTok{color =} \StringTok{"black"}\NormalTok{, }\AttributeTok{size=}\DecValTok{12}\NormalTok{),}
        \AttributeTok{legend.title =} \FunctionTok{element\_blank}\NormalTok{()}
\NormalTok{  )}
\end{Highlighting}
\end{Shaded}

\begin{center}
\pandocbounded{\includegraphics[keepaspectratio]{gridsearch_parkinson_behavioral_analyses_files/figure-pdf/unnamed-chunk-29-3.pdf}}
\end{center}

\begin{Shaded}
\begin{Highlighting}[]
\FunctionTok{ggsave}\NormalTok{(}\StringTok{"plots/bonusround\_chosen\_not\_chosen\_options\_confidence.png"}\NormalTok{, }\AttributeTok{dpi=}\DecValTok{300}\NormalTok{, }\AttributeTok{width=}\DecValTok{7}\NormalTok{, }\AttributeTok{height =} \DecValTok{5}\NormalTok{)}

\CommentTok{\# ggplot(df\_mean\_reward\_subject, aes(x=group, y=mean\_reward, group = group, fill = group)) +}
\CommentTok{\#   \#geom\_hline(data=filter(df\_random\_performance, environment=="Rough"), linetype="dotted", aes(yintercept=z\_learn\_envs)) + }
\CommentTok{\#   \#geom\_hline(data=filter(df\_random\_performance, environment=="Smooth"), linetype="dotted",aes(yintercept=z\_learn\_envs)) +}
\CommentTok{\#   geom\_boxplot(outlier.shape = NA, width = w\_box) +}
\CommentTok{\#   stat\_summary(fun = mean, geom="point", shape = 23, fill = "white", size=1.8) +   }
\CommentTok{\#   geom\_jitter(aes(x = as.numeric(group) +  0.1 + 0.15,  colour=group),  shape = 21, size = 1.3, height = jit\_height, width = jit\_width, alpha = jit\_alpha, colour = "black") +}
\CommentTok{\#   scale\_fill\_manual(values=groupcolors) +}
\CommentTok{\#   \#scale\_color\_viridis(discrete=TRUE, name=\textquotesingle{}Group\textquotesingle{}, direction=1) +}
\CommentTok{\#   \#scale\_fill\_viridis(discrete=TRUE, name=\textquotesingle{}Group\textquotesingle{}, direction=1) +}
\CommentTok{\#   scale\_y\_continuous("Mean reward", limits = c(18, 42), breaks = seq(20,40,5)) +}
\CommentTok{\#   scale\_x\_discrete("") +}
\CommentTok{\#   ggtitle("Performance of groups") +}
\CommentTok{\#   theme\_classic() +}
\CommentTok{\#   theme(\#aspect.ratio = 1,}
\CommentTok{\#         plot.title = element\_text(hjust = 0.5, size=10),}
\CommentTok{\#         legend.title = element\_blank(),}
\CommentTok{\#         legend.position = \textquotesingle{}none\textquotesingle{},}
\CommentTok{\#         legend.text =  element\_text(colour="black"),}
\CommentTok{\#         strip.background = element\_blank(),}
\CommentTok{\#         axis.text = element\_text(colour = "black"),}
\CommentTok{\#         panel.grid.major = element\_blank(),}
\CommentTok{\#         panel.grid.minor = element\_blank())}
\CommentTok{\# }
\CommentTok{\# ggsave("plots/performance\_by\_group.png", dpi=300, height = 3, width = 5 )}
\end{Highlighting}
\end{Shaded}

\chapter{Appendix}\label{appendix}

\section{Distribution of BDI, MMSE, and YH in each
group}\label{distribution-of-bdi-mmse-and-yh-in-each-group}

\begin{Shaded}
\begin{Highlighting}[]
\CommentTok{\# dotplot BDI}
\CommentTok{\# p\_dotplot\_BDI \textless{}{-} }
\FunctionTok{ggplot}\NormalTok{(df\_sample, }\FunctionTok{aes}\NormalTok{(}\AttributeTok{x =}\NormalTok{ BDI, }\AttributeTok{fill =}\NormalTok{ group)) }\SpecialCharTok{+}
  \FunctionTok{facet\_wrap}\NormalTok{(}\SpecialCharTok{\textasciitilde{}}\NormalTok{group) }\SpecialCharTok{+}
  \FunctionTok{geom\_dotplot}\NormalTok{(}\AttributeTok{binwidth =} \DecValTok{1}\NormalTok{, }\AttributeTok{dotsize =} \DecValTok{1}\NormalTok{) }\SpecialCharTok{+}
  \FunctionTok{scale\_fill\_manual}\NormalTok{(}\AttributeTok{values =}\NormalTok{ groupcolors) }\SpecialCharTok{+}
  \FunctionTok{scale\_x\_continuous}\NormalTok{(}\StringTok{"BDI score"}\NormalTok{) }\SpecialCharTok{+} 
  \FunctionTok{scale\_y\_continuous}\NormalTok{(}\ConstantTok{NULL}\NormalTok{, }\AttributeTok{breaks =} \ConstantTok{NULL}\NormalTok{) }\SpecialCharTok{+} 
  \FunctionTok{coord\_fixed}\NormalTok{(}\AttributeTok{ratio =} \DecValTok{15}\NormalTok{) }\SpecialCharTok{+}
  \FunctionTok{theme\_classic}\NormalTok{() }\SpecialCharTok{+}
  \FunctionTok{theme}\NormalTok{(}
    \AttributeTok{legend.title =} \FunctionTok{element\_blank}\NormalTok{(),}
    \AttributeTok{legend.position =} \StringTok{\textquotesingle{}none\textquotesingle{}}\NormalTok{,}
    \AttributeTok{strip.text =} \FunctionTok{element\_text}\NormalTok{(}\AttributeTok{size=}\DecValTok{14}\NormalTok{),}
    \AttributeTok{legend.text =}  \FunctionTok{element\_text}\NormalTok{(}\AttributeTok{colour=}\StringTok{"black"}\NormalTok{),}
    \AttributeTok{text =} \FunctionTok{element\_text}\NormalTok{(}\AttributeTok{colour =} \StringTok{"black"}\NormalTok{),}
    \AttributeTok{strip.background =}\FunctionTok{element\_blank}\NormalTok{(),}
    \AttributeTok{axis.text.x =} \FunctionTok{element\_text}\NormalTok{(}\AttributeTok{colour=}\StringTok{"black"}\NormalTok{),}
    \AttributeTok{axis.text.y =} \FunctionTok{element\_text}\NormalTok{(}\AttributeTok{colour=}\StringTok{"black"}\NormalTok{),}
    \AttributeTok{panel.grid.major =} \FunctionTok{element\_blank}\NormalTok{(),}
    \AttributeTok{panel.grid.minor =} \FunctionTok{element\_blank}\NormalTok{())}
\end{Highlighting}
\end{Shaded}

\begin{center}
\pandocbounded{\includegraphics[keepaspectratio]{gridsearch_parkinson_behavioral_analyses_files/figure-pdf/unnamed-chunk-30-1.pdf}}
\end{center}

\begin{Shaded}
\begin{Highlighting}[]
\CommentTok{\# p\_dotplot\_MMSE \textless{}{-} }
\FunctionTok{ggplot}\NormalTok{(df\_sample, }\FunctionTok{aes}\NormalTok{(}\AttributeTok{x =}\NormalTok{ MMSE, }\AttributeTok{fill =}\NormalTok{ group)) }\SpecialCharTok{+}
  \FunctionTok{facet\_wrap}\NormalTok{(}\SpecialCharTok{\textasciitilde{}}\NormalTok{group) }\SpecialCharTok{+}
  \FunctionTok{geom\_dotplot}\NormalTok{(}\AttributeTok{binwidth =} \DecValTok{1}\NormalTok{, }\AttributeTok{dotsize =} \DecValTok{1}\NormalTok{) }\SpecialCharTok{+}
  \FunctionTok{scale\_fill\_manual}\NormalTok{(}\AttributeTok{values =}\NormalTok{ groupcolors) }\SpecialCharTok{+}
  \FunctionTok{scale\_x\_continuous}\NormalTok{(}\StringTok{"MMSE score"}\NormalTok{) }\SpecialCharTok{+} 
  \FunctionTok{scale\_y\_continuous}\NormalTok{(}\ConstantTok{NULL}\NormalTok{, }\AttributeTok{breaks =} \ConstantTok{NULL}\NormalTok{) }\SpecialCharTok{+} 
  \FunctionTok{coord\_fixed}\NormalTok{(}\AttributeTok{ratio =} \DecValTok{15}\NormalTok{) }\SpecialCharTok{+}
  \FunctionTok{theme\_classic}\NormalTok{() }\SpecialCharTok{+}
  \FunctionTok{theme}\NormalTok{(}\AttributeTok{plot.title =} \FunctionTok{element\_text}\NormalTok{(}\AttributeTok{hjust =} \FloatTok{0.5}\NormalTok{, }\AttributeTok{size =} \DecValTok{10}\NormalTok{),}
        \AttributeTok{legend.title =} \FunctionTok{element\_blank}\NormalTok{(),}
        \AttributeTok{legend.position =} \StringTok{\textquotesingle{}none\textquotesingle{}}\NormalTok{,}
        \AttributeTok{legend.text =}  \FunctionTok{element\_text}\NormalTok{(}\AttributeTok{colour=}\StringTok{"black"}\NormalTok{),}
        \AttributeTok{text =} \FunctionTok{element\_text}\NormalTok{(}\AttributeTok{colour =} \StringTok{"black"}\NormalTok{),}
        \AttributeTok{strip.background =}\FunctionTok{element\_blank}\NormalTok{(),}
        \AttributeTok{axis.text.x =} \FunctionTok{element\_text}\NormalTok{(}\AttributeTok{colour=}\StringTok{"black"}\NormalTok{),}
        \AttributeTok{axis.text.y =} \FunctionTok{element\_text}\NormalTok{(}\AttributeTok{colour=}\StringTok{"black"}\NormalTok{),}
        \AttributeTok{panel.grid.major =} \FunctionTok{element\_blank}\NormalTok{(),}
        \AttributeTok{panel.grid.minor =} \FunctionTok{element\_blank}\NormalTok{())}
\end{Highlighting}
\end{Shaded}

\begin{center}
\pandocbounded{\includegraphics[keepaspectratio]{gridsearch_parkinson_behavioral_analyses_files/figure-pdf/unnamed-chunk-30-2.pdf}}
\end{center}

\begin{Shaded}
\begin{Highlighting}[]
\NormalTok{p\_dotplot\_HY }\OtherTok{\textless{}{-}} \FunctionTok{ggplot}\NormalTok{(}\FunctionTok{filter}\NormalTok{(df\_sample, group }\SpecialCharTok{!=} \StringTok{"PNP"}\NormalTok{), }\FunctionTok{aes}\NormalTok{(}\AttributeTok{x =}\NormalTok{ hoehn\_yahr, }\AttributeTok{fill =}\NormalTok{ group)) }\SpecialCharTok{+}
  \FunctionTok{facet\_wrap}\NormalTok{(}\SpecialCharTok{\textasciitilde{}}\NormalTok{group) }\SpecialCharTok{+}
  \FunctionTok{geom\_dotplot}\NormalTok{(}\AttributeTok{binwidth =} \DecValTok{1}\NormalTok{, }\AttributeTok{dotsize =} \DecValTok{1}\NormalTok{) }\SpecialCharTok{+}
  \FunctionTok{scale\_fill\_manual}\NormalTok{(}\AttributeTok{values =}\NormalTok{ groupcolors) }\SpecialCharTok{+}
  \FunctionTok{scale\_x\_continuous}\NormalTok{(}\StringTok{"Hoehn{-}Yahr score"}\NormalTok{) }\SpecialCharTok{+} 
  \FunctionTok{scale\_y\_continuous}\NormalTok{(}\ConstantTok{NULL}\NormalTok{, }\AttributeTok{breaks =} \ConstantTok{NULL}\NormalTok{) }\SpecialCharTok{+} 
  \FunctionTok{coord\_fixed}\NormalTok{(}\AttributeTok{ratio =} \DecValTok{15}\NormalTok{) }\SpecialCharTok{+}
  \FunctionTok{theme\_classic}\NormalTok{() }\SpecialCharTok{+}
  \FunctionTok{theme}\NormalTok{(}\AttributeTok{plot.title =} \FunctionTok{element\_text}\NormalTok{(}\AttributeTok{hjust =} \FloatTok{0.5}\NormalTok{, }\AttributeTok{size =} \DecValTok{10}\NormalTok{),}
        \AttributeTok{legend.title =} \FunctionTok{element\_blank}\NormalTok{(),}
        \AttributeTok{legend.position =} \StringTok{\textquotesingle{}none\textquotesingle{}}\NormalTok{,}
        \AttributeTok{legend.text =}  \FunctionTok{element\_text}\NormalTok{(}\AttributeTok{colour=}\StringTok{"black"}\NormalTok{),}
        \AttributeTok{text =} \FunctionTok{element\_text}\NormalTok{(}\AttributeTok{colour =} \StringTok{"black"}\NormalTok{),}
        \AttributeTok{strip.background =}\FunctionTok{element\_blank}\NormalTok{(),}
        \AttributeTok{axis.text.x =} \FunctionTok{element\_text}\NormalTok{(}\AttributeTok{colour=}\StringTok{"black"}\NormalTok{),}
        \AttributeTok{axis.text.y =} \FunctionTok{element\_text}\NormalTok{(}\AttributeTok{colour=}\StringTok{"black"}\NormalTok{),}
        \AttributeTok{panel.grid.major =} \FunctionTok{element\_blank}\NormalTok{(),}
        \AttributeTok{panel.grid.minor =} \FunctionTok{element\_blank}\NormalTok{())}
\end{Highlighting}
\end{Shaded}

\section{Performance as function of BDI, MMSE, and
Hoehn-Yahr}\label{performance-as-function-of-bdi-mmse-and-hoehn-yahr}

\subsection{Performance as function of depression score
(BDI)}\label{performance-as-function-of-depression-score-bdi}

The plots show performance as function of depression score (BDI-II),
separately for each group. Opposing trends were found in the different
groups: for patients with polyneuropathy (PNP), there was a negative
relation such that patients with higher depression scores obtained
\emph{lower} rewards. For the two Parkinson groups, the relation was
positive, such that patients reporting more severe symptoms obtained
\emph{higher} rewards.

\begin{center}
\pandocbounded{\includegraphics[keepaspectratio]{gridsearch_parkinson_behavioral_analyses_files/figure-pdf/unnamed-chunk-31-1.pdf}}
\end{center}

\subsection{Performance as function of Mini-Mental State Examination
(MMSE)}\label{performance-as-function-of-mini-mental-state-examination-mmse}

\begin{center}
\pandocbounded{\includegraphics[keepaspectratio]{gridsearch_parkinson_behavioral_analyses_files/figure-pdf/unnamed-chunk-32-1.pdf}}
\end{center}

\subsection{Performance as function of Hoehn-Yahr (Parkinson patients
only)}\label{performance-as-function-of-hoehn-yahr-parkinson-patients-only}

The Hoehn-Yahr scale provides information about the severity of motor
impairments in Parkinson's disease, with higher scores indicating
greater severity.

\begin{center}
\pandocbounded{\includegraphics[keepaspectratio]{gridsearch_parkinson_behavioral_analyses_files/figure-pdf/unnamed-chunk-33-1.pdf}}
\end{center}

\phantomsection\label{refs}
\begin{CSLReferences}{1}{0}
\bibitem[\citeproctext]{ref-abbott2010levodopa}
Abbott, A. (2010). Levodopa: The story so far. \emph{Nature},
\emph{466}(7310), S6--S7.

\bibitem[\citeproctext]{ref-beck1996beck}
Beck, A. T., Steer, R. A., Brown, G. K., et al. (1996). \emph{Beck
depression inventory}.

\bibitem[\citeproctext]{ref-folstein1975mini}
Folstein, M. F., Folstein, S. E., \& McHugh, P. R. (1975).
{``Mini-mental state''}: A practical method for grading the cognitive
state of patients for the clinician. \emph{Journal of Psychiatric
Research}, \emph{12}(3), 189--198.

\bibitem[\citeproctext]{ref-giron2023developmental}
Giron, A. P., Ciranka, S., Schulz, E., Bos, W. van den, Ruggeri, A.,
Meder, B., \& Wu, C. M. (2023). Developmental changes in exploration
resemble stochastic optimization. \emph{Nature Human Behaviour},
\emph{7}(11), 1955--1967.

\bibitem[\citeproctext]{ref-goetz2004movement}
Goetz, C. G., Poewe, W., Rascol, O., Sampaio, C., Stebbins, G. T.,
Counsell, C., Giladi, N., Holloway, R. G., Moore, C. G., Wenning, G. K.,
et al. (2004). Movement disorder society task force report on the hoehn
and yahr staging scale: Status and recommendations the movement disorder
society task force on rating scales for parkinson's disease.
\emph{Movement Disorders}, \emph{19}(9), 1020--1028.

\bibitem[\citeproctext]{ref-hautzinger2006beck}
Hautzinger, M., Keller, F., \& Kühner, C. (2006). \emph{Beck
depressions-inventar (BDI-II)}. Harcourt Test Services.

\bibitem[\citeproctext]{ref-hoehn1967parkinsonism}
Hoehn, M. M., \& Yahr, M. D. (1967). Parkinsonism: Onset, progression,
and mortality. \emph{Neurology}, \emph{17}(5), 427--427.

\bibitem[\citeproctext]{ref-Meder2021_ExplorationChildren}
Meder, B., Wu, C. M., Schulz, E., \& Ruggeri, A. (2021). Development of
directed and random exploration in children. \emph{Developmental
Science}, \emph{24}(4), e13095.
https://doi.org/\url{https://doi.org/10.1111/desc.13095}

\bibitem[\citeproctext]{ref-sadeghiyeh2020temporal}
Sadeghiyeh, H., Wang, S., Alberhasky, M. R., Kyllo, H. M., Shenhav, A.,
\& Wilson, R. C. (2020). Temporal discounting correlates with directed
exploration but not with random exploration. \emph{Scientific Reports},
\emph{10}(1), 4020.

\bibitem[\citeproctext]{ref-Schulz:2019kwg}
Schulz, E., Wu, C. M., Ruggeri, A., \& Meder, B. (2019). Searching for
rewards like a child means less generalization and more directed
exploration. \emph{Psychological Science}, \emph{30}(11), 1561--1572.
\url{https://doi.org/10.1177/0956797619863663}

\bibitem[\citeproctext]{ref-tambasco2018levodopa}
Tambasco, N., Romoli, M., \& Calabresi, P. (2018). Levodopa in
parkinson's disease: Current status and future developments.
\emph{Current Neuropharmacology}, \emph{16}(8), 1239--1252.

\bibitem[\citeproctext]{ref-Wu_2018grid}
Wu, C. M., Schulz, E., Speekenbrink, M., Nelson, J. D., \& Meder, B.
(2018). {Generalization guides human exploration in vast decision
spaces}. \emph{Nature Human Behaviour}, \emph{2}, 915--924.
\url{https://doi.org/10.1038/s41562-018-0467-4}

\end{CSLReferences}




\end{document}
